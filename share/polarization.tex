\lab{Polarization}\label{lab:polarization}

\apparatus
\equipn{laser}{1/group}
\equipn{calcite crystal (flattest available)}{1/group}
\equipn{polarizing films}{2/group}
\equipn{Na gas discharge tube}{1/group}
\equipn{photovoltaic cell and collimator}{1/group}

\begin{goals}

\item[] Make qualitative observations about the polarization of light.

\item[] Test quantitatively the hypothesis that polarization
relates to the direction of the field vectors in an
electromagnetic wave.
\end{goals}

\introduction

 It's common knowledge that there's more to light than meets
the eye: everyone has heard of infrared and ultraviolet
light, which are visible to some other animals but not to
us. Another invisible feature of the wave nature of light is
far less well known. Electromagnetic waves are transverse,
i.e., the electric and magnetic field vectors vibrate in
directions perpendicular to the direction of motion of the
wave. Two electromagnetic waves with the same wavelength can
therefore be physically distinguishable, if their electric
and magnetic fields are twisted around in different
directions. Waves that differ in this way are said to have
different polarizations.

\figcaption{op-pol-emwave}{An electromagnetic wave has electric
and magnetic field vectors that vibrate in the directions
perpendicular to its direction of motion. The wave's direction
of polarization is defined as the line along which the electric
field lies.}

Maybe we polarization-blind humans are missing out on
something. Some fish, insects, and crustaceans can detect
polarization. Most sources of
visible light (such as the sun or a light bulb) are
unpolarized. An unpolarized beam of light contains a random
mixture of waves with many different directions of
polarization, all of them changing from moment to moment,
and from point to point within the beam.

\section{Qualitative Observations}

Before doing anything else, turn on your gas discharge tube,
so it will be warmed up when you are ready to do part E.

\labpart{ Double refraction in calcite}

Place a calcite crystal on this page.  You will see two
images of the print through the crystal.

To understand why this happens, try shining the laser beam
on a piece of paper and then inserting the calcite crystal
in the beam. If you rotate the crystal around in different
directions, you should be able to get two distinct spots to
show up on the paper. (This may take a little trial and
error, partly because the effect depends on the correct
orientation of the crystal, but also because the crystals
are not perfect, and it can be hard to find a nice smooth
spot through which to shine the beam.)

In the refraction lab, you've already seen how a beam of
light can be bent as it passes through the interface between
two media. The present situation is similar because the
laser beam passes in through one face of the crystal and
then emerges from a parallel face at the back. You have
already seen that in this type of situation, when the beam
emerges again, its direction is bent back parallel to its
original direction, but the beam is offset a little bit.
What is different here is that the same laser beam splits up
into two parts, which bumped off course by different amounts.

What's happening is that calcite, unlike most substances,
has a different index of refraction depending on the
polarization of the light. Light travels at a different
speed through calcite depending on how the electric and
magnetic fields are oriented compared to the crystal. The
atoms inside the crystal are packed in a three-dimensional
pattern sort of like a stack of oranges or cannonballs. This
packing arrangement has a special axis of symmetry, and
light polarized along that axis moves at one speed, while
light polarized perpendicular to that axis moves at a different speed.

It makes sense that if the original laser beam was a random
mixture of all possible directions of polarization, then
each part would be refracted by a different amount. What is
a little more surprising is that two separated beams emerge,
with nothing in between. The incoming light was composed of
light with every possible direction of polarization. You
would therefore expect that the part of the incoming light
polarized at, say, 45\degunit compared to the crystal's axis
would be refracted by an intermediate amount, but that
doesn't happen. This surprising observation, and all other
polarization phenomena, can be understood based on the
vector nature of electric and magnetic fields, and the
purpose of this lab is to lead you through a series of
observations to help you understand what's really going on.

\labpart{ A polarized beam entering the calcite}

\figcaption{op-pol-splitrays}{A single laser beam entering a calcite
crystal breaks up into two parts, which are
refracted by different amounts.}
\figcaption{op-pol-splitwave}{The calcite splits the wave into two parts, polarized
in perpendicular directions compared to each other.}
\fig{op-pol-mixtures}

We need not be restricted to speculation about what was
happening to the part of the light that entered the calcite
crystal polarized at a 45\degunit angle. You can use a
polarizing film, often referred to informally as a
``Polaroid,'' to change unpolarized light into a beam of
only one specific polarization. In this part of the lab, you
will use a polarizing film to produce a beam of light
polarized at a 45\degunit angle to the crystal's internal axis.

   If you simply look through the film, it doesn't look like
anything special --- everything just looks dimmer, like
looking through sunglasses. The light reaching your eye is
polarized, but your eye can't tell that. If you looked at
the film under a microscope, you'd see a pattern of stripes,
which select only one direction of polarization of the light
that passes through.

Now try interposing the film between the laser and the
crystal. The beam reaching the crystal is now polarized
along some specific direction. If you rotate the film, you
change beam's direction of polarization. If you try various
orientations, you will be able to find one that makes one of
the spots disappear, and another orientation of the film, at
a 90\degunit angle compared to the first, that makes the
other spot go away. When you hold the film in one of these
directions, you are sending a beam into the crystal that is
either purely polarized along the crystal's axis or purely
polarized at 90\degunit to the axis.

By now you have already seen what happens if the film is at
an intermediate angle such as 45\degunit. Two spots appear
on the paper in the same places produced by an unpolarized
source of light, not just a single spot at the midpoint.
This shows that the crystal is not just throwing away the
parts of the light that are out of alignment with its axis.
What is happening instead is that the crystal will accept a
beam of light with any polarization whatsoever, and split it
into two beams polarized at 0 and 90\degunit compared
to the crystal's axis.

This behavior actually makes sense in terms of the wave
theory of light. Light waves are supposed to obey the
principle of superposition, which says that waves that pass
through each other add on to each other. A light wave is
made of electric and magnetic fields, which are vectors, so
it is vector addition we're talking about in this case. A
vector at a 45\degunit angle can be produced by adding two
perpendicular vectors of equal length. The crystal 
therefore cannot respond any differently to 45-degree
polarized light than it would to a 50-50 mixture of  light
with 0-degree and 90-degree polarization.

\figcaption{op-pol-90degsuperpos}{The principle of superposition implies that if the 0\degunit
and 90\degunit polarizations produce two different spots,
then the two waves superimposed must produce
those two spots, not a single spot at an intermediate
location.}

\labpart{ Two polarizing films}

So far I've just described the polarizing film as a device
for producing polarized light. But one can apply to the
polarizing film the same logic of superposition and vector
addition that worked with the calcite crystal. It would not
make sense for the film simply to throw away any waves that
were not perfectly aligned with it, because a field oriented
on a slant can be analyzed into two vector components, at 0
and 90\degunit with respect to the film. Even if one
component is entirely absorbed, the other component should
still be transmitted.

\fig{op-pol-crossedpolaroids}

Based on these considerations, now think about what will
happen if you look through two polarizing films at an angle
to each other, as shown in the figure above. \emph{Do not look
into the laser beam!\/} Just look around the room. What will
happen as you change the angle $\theta $?


\labpart{ Three polarizing films}

\fig{op-pol-3polaroids}

Now suppose you start with two films at a 90\degunit angle
to each other, and then sandwich a third film between them
at a 45\degunit angle, as shown in the two figures above.
Make a prediction about what will happen, and discuss your
prediction with your instructor before you make the actual observation.

\section{Quantitative Observations}

\labpart{ Intensity of light passing through two polarizing films}

In this part of the lab, you will make numerical measurements
of the transmission of initially unpolarized light
transmitted through two polarizing films at an angle $\theta
$ to each other. To measure the intensity of the light that
gets through, you will use a photocell, which is a device
that converts light energy into an electric current.

You will use a voltmeter to measure the voltage across the
photocell when light is shining on it. A photovoltaic cell is
a complicated nonlinear device, but I've found empirically
that under the conditions we're using in this experiment,
the voltage is proportional to the power of the light striking
the cell: twice as much light results in twice the voltage.

This measurement requires a source of light that is
unpolarized, constant in intensity, and comes from a
specific direction so it can't get to the photocell without
going through the polaroids. The ambient light in the room
is nearly unpolarized, but varies randomly as people walk in
front of the light fixtures, etc. The laser beam is constant
in intensity, but as I was creating this lab I found to my
surprise that it is partially polarized, with a polarization
that varies over time. A more suitable source of light is the
sodium gas discharge tube, which makes a nearly monochromatic,
unpolarized yellow light. Make sure you have allowed it to
warm up for at least 15-20 minutes before using it; before
it warms up, it makes a reddish light, and the polaroids do
not work very well on that color.

Make measurements of the relative intensity of light
transmitted through the two polarizing films, using a
variety of angles $\theta $. Don't assume that the notches
on the plastic housing of the polarizing films are a good
indication of the orientation of the films themselves.

\prelab

\prelabquestion Given the angle $\theta$ between the polarizing films, predict the 
ratio $|\vc{E}'|/|\vc{E}|$ of the transmitted electric field to the incident
electric field.

\prelabquestion Based on your answer to P1, predict the ratio $P'/P$ of the transmitted
power to the incident power.

\prelabquestion Sketch a graph of your answer to P2. Superimposed on the same graph, show
a qualitative prediction of how it would change if the polaroids were not
100\% perfect at filtering out one component of the field.

\analysis

Discuss your qualitative results in terms of superposition
and vector addition.

Graph your results from part E, and superimpose a
theoretical curve for comparison. Discuss how your results
compare with theory. Since your measurements of light
intensity are relative, just scale the theoretical curve so
that its maximum matches that of the experimental data. (You
might think of comparing the intensity transmitted through
the two polaroids with the intensity that you get with no
polaroids in the way at all. This doesn't really work,
however, because in addition to acting as polarizers, the
polaroids simply absorb a certain percentage of the light,
just as any transparent material would.)
