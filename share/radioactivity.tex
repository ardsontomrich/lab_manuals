\lab{Radioactivity}\label{lab:radioactivity}

Note to the lab technician: The isotope generator kits came with 250 mL bottles
of eluting solution (0.9\% NaCl in 0.04M HCl, made with deionized water). If we
ever run out of the solution, we can make more from materials in the chem
stockroom.  The GM counters have 9 V batteries, which should be checked before
lab.

\apparatus
\equip{isotope generator kit}
\equip{Geiger-M\"{u}ller (GM) counter}
\equip{computer with Logger Pro software and LabPro interface}
\equip{``grabber'' clamp and stand}
\equip{wood blocks, 25 mm thick}
\equip{pieces of steel, 17 mm thick}

\goal{Determine the properties of an unknown radioactive source.}

\introduction

You're a science major, but even if you weren't, it would be
important for you as a citizen and a voter to understand
the properties of radiation. As an example of an important social
issue, many environmentalists
who had previously opposed nuclear power now believe that its
benefits, due to reduction of global warming, outweigh its
problems, such as disposal of waste. To understand such
issues, you need to learn to reason about radioactivity
quantitatively.

A radioactive substance contains atoms whose nuclei spontaneously
decay into nuclei of a different type. Nobody has ever succeeded
in finding a physical law that would predict when a particular
nucleus will ``choose'' to decay. The process is random, but we
can make quantitative statements about how quickly the process tends
to happen.
A radioactive substance has a certain \emph{half-life}, defined
as the time required before (on the average) 50\%
of its nuclei will have decayed.

%  calc -e "e=1.6 10^-19 C; t=(3)(3600 s); f=.1; m=50 kg; P=(10 10^-6)(3.7 10^10 s-1)(.66 GV)(e); E=Ptf; Gy=E/m; rad=100Gy; Q=1; N=.1; muSv=(Gy)(QN)(10^6)"
%  result is 80 microSv=8 mrem
%  for numerical data on hormesis, http://www.radpro.com/641luckey.pdf
%  pregnancy: http://www.acog.org/from_home/publications/green_journal/2004/v103n6p1326.pdf
%  See meki lab notes for more detailed info, with lots of references. I have some relevant papers in ~/Papers.

\section*{Safety}

\newcommand{\mus}{$\mu$Sv}

The radioactive source used in this lab is very
weak. It is so weak that it is exempt from government
regulation, it can be sent in the mail, and if the college bought new equipment, it would
be legal to throw out the old sources in the trash.
The following table compares some radiation doses, including an estimate of the typical dose
you might receive in this lab. These are in units of microSieverts (\mus, 10 \mus=1 millirem).


\begin{tabular}{p{50mm}r}
  CT scan                                    & $\sim$ 10,000 \mus \\
  natural background per year                & 2,000-7,000 \mus \\
  health guidelines for exposure to a fetus  & 1,000 \mus \\
  flying from New York to Tokyo              & 150 \mus \\
  this lab                                   & $\sim$ 80 \mus \\
  chest x-ray                                & 50 \mus
\end{tabular}

A variety of experiments show cases in which
low levels of radiation activate cellular damage control mechanisms, increasing
the health of the organism. For example, there is evidence exposure to radiation up to a certain level makes mice grow
faster; makes guinea pigs' immune systems function better against diphtheria; increases fertility in female humans, trout, and mice;
improves fetal mice's resistance to disease; reduces genetic abnormalities in humans;
increases the life-spans of flour beetles and mice; and reduces mortality from cancer in mice and humans.
This type of effect is called radiation
hormesis. Nobody knows for sure, but it's possible that you will receive a very tiny improvement in your health from the radiation exposure
you experience in lab today.

Although low doses of radiation may be beneficial, 
governments, employers, and schools generally practice a philosophy called
ALARA, which means to make radiation doses As Low As Reasonably Achievable.
You should adhere to this approach in this lab.
In general, internal exposure to radiation produces more of an effect than external
exposure, so you should not eat or drink during this lab, and
you should avoid getting any of the radioactive substances in an
open cut. You should also reduce your exposure by not spending an unnecessarily large amount of time
with your body very close to the source, e.g., you should not hold it in the palm of your hand
for the entire lab period.

\section*{The source and the GM counter}

You are supplied with a radioactive source packaged inside
a small plastic disk about the size of the spindle that
fits inside a roll of scotch tape.

Our radiation detector for this lab is called a Geiger-M\"{u}ller (GM) counter.
It is is a cylinder full of gas, with the outside
of the cylinder at a certain voltage and a wire running down
its axis at another voltage. The voltage difference creates
a strong electric field. When ionizing radiation enters
the cylinder, it can ionize the gas, separating negatively
charged ions (electrons) and positively charged ones
(atoms lacking some electrons). The electric field accelerates
the ions, making them hit other air molecules, and causing
a cascade of ions strong enough to be measured as an electric
current.

If you look at the top side of the GM counter (behind the top of the front panel),
you'll see a small window. Non-penetrating radiation can only get in through
this thin layer of mica. (Gammas can go right in through the plastic housing.)

\observations

\labpart{Background}


Use the GM counter to observe the background radiation in the room. This radiation is
probably a combination of gamma rays from naturally occurring minerals
in the ground plus betas and gammas from building materials such 
as concrete. If you like, you can walk around the room and see if you can detect
any variations in the intensity of the background.

Estimate the rate at which the GM detector counts when it is exposed
only to background. Starting at this point, it is more convenient to
interface the GM counter to the computer. Plug the cable into
into DIG/SONIC 1 on the LabPro interface. Start LoggerPro 3 on the
computer, and open the file Probes and Sensors : Radiation monitor : Counts versus time.
The interface is not able to automatically identify this particular sensor, so
the software will ask you to confirm that you really do have this type of sensor
hooked up; confirm this by clicking on Connect.

Now when you hit the Collect button, the sensor will start graphing the number of counts
it receives during successive 5-second intervals. The y axis of the graph is counts
per 5 seconds, and the x axis is time.

Once you've made some preliminary observations, try to get a good measurement of
the background, counting for several minutes so as to reduce the statistical errors.
It is important to get a good measurement of the background rate, because in later
parts of the lab you'll need to subtract it from all the other count rates you measure.
For longer runs like this, it is convenient to let the software collect data for about a minute
at a time, rather than 5 seconds. To get it to do this, do Experiment : Data collection : 60 seconds/sample.

\labpart{Type of radiation}

Your first task in the lab is to figure out whether the source emits
alpha, beta, or gamma radiation --- or perhaps some mixture of these.
It is up to you to decide how to do this, but essentially you want to
use the fact that they are absorbed differently in matter; referring
to your textbook, you'll see that the historical labels $\alpha$, $\beta$, and $\gamma$ were assigned
purely on the basis of the differences in absorption, before anyone even knew what they
really were. Note that the fact that you were able to detect the radiation
at all in part A implies that the radiation was able to penetrate the thin
plastic walls of the source. Also keep in mind that the descriptions of absorption
in a textbook are generalizations that do not take into account the energy of the
particles. For example, low-energy betas could be absorbed by a kleenex, whereas
high-energy betas could penetrate cardboard.

There are six things you could try to prove through your measurements:
\begin{enumerate}
\item The source inside the plastic container emits alphas.
\item The source does not emit alphas.
\item The source emits betas.
\item The source does not emit betas.
\item The source emits gammas.
\item The source does not emit gammas.
\end{enumerate}

Try to figure out which of these six statements you can either definitively prove or
definitively disprove. Because you are not allowed to extract the source from the packaging,
there will be some cases in which you cannot draw any definitive conclusion one way or the other.
You may find it convenient to switch back to a shorter time per sample, such as 10 seconds/sample.

\labpart{Distance dependence}

Measure the count rate at several different distances from the source.
The goal is to find the mathematical form of this function (see Analysis, below).
Distances of less than about 10 cm do not work well, because the size of the GM tube
is comparable to 10 cm.

\labpart{Absorption}

Only odd-numbered lab groups should do part D.

Measure the reduction in count rate when a 25 mm thick wood block is 
interposed between the source and the detector, and likewise for 17 mm of steel.
Keep the distance from the source to the detector constant throughout.
Let the counter run for at least five minutes each time.
Based on these observations, predict the count rate you would
get with two 17-mm thicknesses of steel instead of one. Test your prediction.

\labpart{Scattering}

Only even-numbered lab groups should do part E.

\fig{mo-rad-scattering}

In addition to being absorbed, radioactivity can be \emph{scattered}, i.e., turned off
course without being stopped completely. The figure shows a possible way of detecting
scattering. The steel is not between the source and the detector, so absorption will
have no effect. But if scattering occurs, it is possible that the count rate in the
detector will \emph{increase} when the steel is inserted, because some particles that
would have missed the detector will be scattered into it by the steel. Try to detect
whether there is any measurable scattering. To get the best chance of detecting it,
you want the whole setup to squeezed into as tight a space as possible. You will need
to count for a fairly long time, probably five minutes or more.

\labpart{Decay curve}

\newcommand{\ces}{$^{NNN}\zu{Cs}$}

The source consists of a particular isotope of cesium; we'll refer to it as \ces, since the main
goal of this lab is to determine what the unknown isotope actually is. It decays to an isotope
of barium, and rather than decaying to the ground state of the barium nucleus, it nearly always
decays to an excited state, which then emits the radiation you characterized in part B.
Although the half-life of the parent cesium isotope is many years, the half-life of the
excited isotope in the barium daughter is short enough that it can be observed during the lab
period. However, if the cesium and barium are not separated, then no time variation will
be observed, because the supply of barium nuclei is being continuously replenished by
decay of the parent cesium.

\fig{mo-rad-levels}

To get around this, the source is packaged so that when a weak acid solution is forced through
it, a small amount of barium is washed out. Note that the yellow tape around the circumference
of the source has an arrow on it. This arrow points in the direction that you're supposed
to make the solution flow. The isotope generator kit has coin-sized steel trays on which to
collect a few drops of the radioactive solution.

Use the syringe to draw 1 mL of the acid solution from the 250-mL bottle (labeled ``eluting solution'').
Take one of the tiny coin-sized steel trays out of the isotope generator kit and lay it on the lab bench.
Remove the little plugs from the top and bottom of the radioactive source. Stick the syringe in the
in-flow hole, and use the plunger to force seven drops of liquid out onto the tray. Note that
the amount of liquid that flows from the syringe into the source is quite a bit more than the
amount that comes out into the tray. If you have solution left in the syringe at the end of lab,
squeeze it back out into the 250-mL bottle.

Use the computer to collect data on the rate of decay as a function of time. About 5 or 10 seconds
per sample works well.

When you're done, make sure to shut off the GM counter so that its battery doesn't get drained.

\section*{Waste disposal}

To get an idea of what a non-issue  radioactive waste disposal is in this lab,
recall that it would be legal to throw the \emph{entire} source in the trash --- although we won't
actually do that.
The amount of radioactive material that you wash out in part E is a tiny fraction of this.
Furthermore, essentially all the radioactivity is
gone by the end of lab. It is therefore not a problem to dump your seven drops of material
down the drain at the end of class.

There is also no chemical disposal issue with this tiny amount of solution.
It's a few drops of very dilute acid, equivalent to a little spritz of lemon juice.

\analysis

In part C, you should first subtract the background
rate from each datum. Then make a log-log plot as described in appendix \ref{appendix:powerlaws},
and see if you can successfully describe the data using a power law.
Note that, just like a human, the GM counter cannot count faster than a certain rate.
This is because every time it gets a count it completely discharges its voltage, and
then it has to recharge itself again. For this reason, it is possible that your data
from very small distances will not agree with the behavior of the data at larger distances.
The documentation for these GM counters says that they can count at up to about
3500 counts per second; this is only a very rough guide, but it gives you some idea what
count rates should be expected to start diverging from ideal behavior.

Estimate the half-lives of any isotopes present in the data. If you find that only one
half-life is present, you can simply determine the 
amount of time required for the count rate to fall off by a factor of two.
If the natural background count rate measured in part A is significant, you will need to subtract it from the raw data.
If more than one half-life is present, try plotting the logarithm of the count rate
as a function of time, and seeing if there are linear sections on the graph.
Note that this is all referring to the half-lives of any decay chain that occurs
\emph{after} the cesium decays to barium. The half-life of the cesium parent
nucleus is much longer, and is not measured in this experiment.

Consult the Wikipedia article ``Isotopes of cesium.'' It has an extremely lengthy
table of all the known isotopes from very light ones (with far too few neutrons to
be stable) to very heavy ones (with far too many). Since the source was shipped to us
through the mail, and sits on the shelf in the physics stockroom for semester after
semester, you can tell that the half-life of the cesium isotope must be fairly long ---
at least on the order years, not months. From this information, you should be able
to narrow down the range of possibilities. (Half-lives in units of years are listed with
``a,'' for ``annum,'' as the unit of time. The notations m1 and m2 mean energy states
that are above the lowest-energy state.) The radioactive isotopes from this remaining
list of possibilities all have their own Wikipedia articles, and these articles give
the properties of the daughter nuclei (isotopes of barium and xenon), including the half-lives
of any gamma-emitting states. Look for one that has a half-life that seems to match
the one you measured in lab. Having tentatively identified the unknown isotope as
this isotope of cesium, check against the results of part B, where you determined
the type of radiation emitted.

\prelab

\prelabquestion
Suppose that in part C you obtain the following data:

\begin{tabular}{ll}
r (cm) & count rate (counts/2 min) \\
10      & 707 \\
20     & 207 \\
30     & 95
\end{tabular}

Suppose that the background rate you measure in part A
is 30 counts per 2 min. Use the technique
described in appendix \ref{appendix:powerlaws} to see if the data can be described
by a power law, and if so, determine the exponent.

\prelabquestion
If a source emits gamma or beta radiation, then the radiation spreads out in all
directions, like an expanding sphere. Based on the scaling of a sphere's surface
area with increasing radius, how would you theoretically expect the
intensity of the radiation to fall off with distance? Would it be a power law?
If so, what power? Why would you \emph{not} expect the same behavior for a source
emitting alpha particles into air?
