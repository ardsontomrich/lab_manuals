\lab{RC Circuits}\label{lab:rc}

\apparatus
\equipn{oscilloscope}{1/group}
\equip{Pasco PI-9587C function generator}{1/group}
\equipn{unknown capacitor}{1/group}
\equipn{known capacitors, 0.05 $\mu$F }{1/group}
\equip{resistors of various values}

\begin{goals}

\item[] Observe the exponential curve of a discharging capacitor.

\item[] Determine the capacitance of an unknown capacitor.
\end{goals}

\introduction

God bless the struggling high school math teacher, but some
of them seem to have a talent for making interesting and
useful ideas seem dull and useless. On certain topics such
as the exponential function, ex, the percentage of students
who figure out from their teacher's explanation what it
really means and why they should care approaches zero.
That's a shame, because there are so many cases where it's
useful. The graphs show just a few of the important
situations in which this function shows up.

\fig{em-rcc-exponentials}

The credit card example is of the form
\begin{equation*}
    y=ae^{t/k}   \qquad   ,
\end{equation*}
while the Chernobyl graph is like
\begin{equation*}
    y=ae^{-t/k}    \qquad   ,
\end{equation*}
In both cases, $e$ is the constant $2.718\ldots$, and $k$ is a
positive constant with units of time, referred to as the
time constant. The first type of equation is referred to as
exponential growth, and the second as exponential decay. The
significance of $k$ is that it tells you how long it takes
for $y$ to change by a factor of $e$. For instance, an 18\%
interest rate on your credit card converts to $k=6.0$ years.
That means that if your credit card balance is \$1000 in
1996, by 2002 it will be \$2718, assuming you never really
start paying down the principal.

An important fact about the exponential function is that it
never actually becomes zero --- it only gets closer and
closer to zero. For instance, the radioactivity near
Chernobyl will never ever become exactly zero. After a while
it will just get too small to pose any health risk, and at
some later time it will get too small to measure with
practical measuring devices.

Why is the exponential function so ubiquitous? Because it
occurs whenever a variable's rate of change is proportional
to the variable itself. In the credit card and Chernobyl examples,
\begin{gather*}
    (\text{rate of increase of credit card debt}) \\
    \qquad	\propto(\text{present credit card debt})    \\
    (\text{rate of decrease of the number of radioactive atoms}) \\
    \qquad	\propto(\text{present number of radioactive atoms})    
\end{gather*}

For the credit card, the proportionality occurs because your
interest payment is proportional to how much you currently
owe. In the case of radioactive decay, there is a proportionality
because fewer remaining atoms means fewer atoms available to
decay and release radioactive particles. This line of thought
leads to an explanation of what's so special about the
constant $e$. If the rate of increase of a variable $y$ is
proportional to $y$, then the time constant $k$ equals one
over the proportionality constant, and this is true only if
the base of the exponential is $e$, not 10 or some other number.

Exponential growth or decay can occur in circuits containing
resistors and capacitors. Resistors and capacitors are the
most common, inexpensive, and simple electrical components.
If you open up a cell phone or a stereo, the vast majority
of the parts you see inside are resistors and capacitors.
Indeed, many useful circuits, known as RC circuits, can be
built out of nothing but resistors and capacitors. In this
lab, you will study the exponential decay of the simplest
possible RC circuit, shown below, consisting of one resistor
and one capacitor in series.

\fig{em-rcc-simplified}

Suppose we initially charge up the capacitor, making an
excess of positive charge on one plate and an excess of
negative on the other. Since a capacitor behaves like
$V=Q/C$, this creates a voltage difference across the
capacitor, and by Kirchoff's loop rule there must be a
voltage drop of equal magnitude across the resistor. By
Ohm's law, a current $I=V/R=Q/RC$ will flow through
the resistor, and we have therefore established a proportionality,
\begin{gather*}
    (\text{rate of decrease of charge on capacitor}) \\
    \qquad	\propto(\text{present charge on capacitor})  \qquad .  \\
\end{gather*}

It follows that the charge on the capacitor will decay
exponentially. Furthermore, since the proportionality
constant is $1/RC$, we find that the time constant of the
decay equals the product of $R$ and $C$. (It may not be
immediately obvious that Ohms times Farads equals seconds, but it does.)

Note that even if we put the charge on the capacitor very
suddenly, the discharging process still occurs at the same
rate, characterized by $RC$. Thus RC circuits can be used to
filter out rapidly varying electrical signals while
accepting more slowly varying ones. A classic example occurs
in stereo speakers. If you pull the front panel off of the
wooden box that we refer to as ``a speaker,'' you will find
that there are actually \emph{two} speakers inside, a small
one for reproducing high frequencies and a large one for the
low notes. The small one, called the tweeter, not only
cannot produce low frequencies but would actually be damaged
by attempting to accept them. It therefore has a capacitor
wired in series with its own resistance, forming an RC
circuit that filters out the low frequencies while
permitting the highs to go through. This is known as a
high-pass filter. A slightly different arrangement of
resistors and inductors is used to make a low-pass filter
to protect the other speaker, the woofer, from high frequencies. 

\observations


In typical filtering applications, the RC time constant is
of the same order of magnitude as the period of a sound
vibration, say $\sim1$ ms. It is therefore necessary to
observe the changing voltages with an oscilloscope rather
than a multimeter. The oscilloscope needs a repetitive
signal, and it is not possible for you to insert and remove
a battery in the circuit hundreds of times a second, so you
will use a function generator to produce a voltage that
becomes positive and negative in a repetitive pattern. Such
a wave pattern is known as a square wave. The mathematical
discussion above referred to the exponential decay of the
charge on the capacitor, but an oscilloscope actually
measures voltage, not charge. As shown in the graphs below,
the resulting voltage patterns simply look like a chain of
exponential curves strung together.

\fig{em-rcc-actualcircuit}

\fig{em-rcc-rcgraphs}

Make sure that the yellow or red ``VAR'' knob, on the front of the
knob that selects the time scale, is clicked into place, not
in the range where it moves freely --- otherwise the times
on the scope are not calibrated.

\labpart{ Preliminary observations}

Pick a resistor and capacitor with a combined RC time
constant of $\sim1$ ms. Make sure the resistor is at least
$\sim10\zu{k}\Omega $, so that the internal resistance of the
function generator is negligible compared to the resistance you supply.

Note that the capacitance values printed on the sides of
capacitors often violate the normal SI conventions about
prefixes. If just a number is given on the capacitor with no
units, the implied units are microfarads, mF. Units of nF
are avoided by the manufacturers in favor of fractional
microfarads, e.g., instead of 1 nF, they would use ``0.001,''
meaning 0.001 $\mu $F. For picofarads, a capital P is
used, ``PF,'' instead of the standard SI ``pF.''

 Use the oscilloscope to observe what happens to the
voltages across the resistor and capacitor as the function
generator's voltage flips back and forth. Note that the
oscilloscope is simply a fancy voltmeter, so you connect it
to the circuit the same way you would a voltmeter, in
parallel with the component you're interested in. Make sure the
scope is set on DC, not AC, by doing CH 1$>$Coupling$>$DC. A
complication is added by the fact that the scope and the
function generator are fussy about having the grounded sides
of their circuits connected to each other. The banana-to-BNC
converter that goes on the input of the scope has a small
tab on one side marked ``GND.'' This side of the scope's
circuit must be connected to the ``LO'' terminal of the
function generator. This means that when you want to switch
from measuring the capacitor's voltage to measuring the
resistor's, you will need to rearrange the circuit a little.

If the trace on the oscilloscope does not look like the one
shown above, it may be because the function generator is
flip-flopping too rapidly or too slowly. The function
generator's frequency has no effect on the RC time constant,
which is just a property of the resistor and the capacitor. 

If you think you have a working setup, observe the effect of
temporarily placing a second capacitor in parallel with the
first capacitor. If your setup is working, the exponential
decay on the scope should become more gradual because you
have increased RC. If you don't see any effect, it probably
means you're measuring behavior coming from the internal $R$
and $C$ of the function generator and the scope.

Use the scope to determine the RC time constant, and
check that it is correct.
Rather than reading times and voltages by eye from the scope's x-y grid, you can make
the scope give you a measuring cursor. Do Cursor$>$Type$>$Time, and Source$>$CH 1 . Use the top
left knob to move the cursor to different times.


\labpart{ Unknown capacitor}

Build a similar circuit using your unknown capacitor plus a
known resistor. Use the unknown capacitor with the same
number as your group number. Take the data you will need in
order to determine the RC time constant, and thus the
unknown capacitance.

As a check on your result, obtain a known capacitor with a
value similar to the one you have determined for your
unknown, and see if you get nearly the same curve on the
scope if you replace the unknown capacitor with the new one.

\prelab

\prelabquestion  Plan how you will determine the capacitance and what
data you will need to take.

\analysis

Determine the capacitance, with error analysis (appendices 2 and 3).
