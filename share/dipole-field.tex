\lab{The Dipole Field}\label{dipolefieldlab}\label{lab:dipole-field}

\widefigcaption{em-dip-fieldvsdistance}{Measuring the variation of the
bar magnet's field with respect to distance.}\label{fig:dipbofr}

\apparatus
\equip{bar magnet}
\equip{compass}
\equip{graph paper, with 1 cm squares}
\equip{Hall effect magnetic field probes}
\equip{LabPro interfaces, DC power supplies, and USB cables}

\goal{Find how the magnetic field of a bar magnet changes with
distance along one of the magnet's lines of symmetry.}

\introduction

This lab is designed to be used along with section 10.3 of Simple
Nature, which is about the superposition (i.e., addition) of fields. That section
is about electric fields, and the basic principle is that if we have
two sets of sources (charges) that would individually create
fields $\zb{E}_1$ and $\zb{E}_2$, then their combined field is
the vector sum $\zb{E}_1+\zb{E}_2$. Static electric fields, however,
are difficult to control and measure. Magnetic fields are much easier
to work with, and the same vector addition principle applies to them.
In this lab, you'll expose a magnetic compass to the superposed magnetic
fields of the earth and a bar magnet. 

\labpart{Qualitative Mapping of the Dipole's Field}
You can use a compass to map out part of the magnetic
field of a bar magnet. It turns out that the bar magnet is the
magnetic equivalent of an electric dipole. The compass is affected by
both the earth's field and the bar magnet's field, and
points in the direction of their vector sum, but if you put the
compass within a few cm of the bar magnet, you're seeing mostly
its field, not the earth's. Investigate the bar magnet's field,
and sketch it in your lab notebook. You should see that it looks like
the field a dipole.

\labpart{Variation of Field With Distance: Deflection of a Magnetic Compass}\label{dipole-compass}
Magnetic fields are actually measured in units of Tesla (T), but
for the purposes of this part of the lab, we'll just measure the fields
in units of the earth's magnetic field. That is, we define the
earth's magnetic field to have a strength of exactly 1.0 in
Fullerton.\footnote{Actually we're defining its horizontal component
to be one unit --- the compass can't respond to vertical fields.
The dip angle of the magnetic field in Fullerton is fairly steep.}
You can infer the strength of the bar magnet's field at a
given point by putting the compass there and seeing how
much it is deflected. The standard notation for magnetic field is
$B$, so we can notate the fields of the earth and the magnet as
$B_e$ and $B_m$. 

The task can be simplified quite a bit if you restrict
yourself to measuring the magnetic field at points along one
of the magnet's two lines of symmetry, shown in the figure. 

\fig{em-dip-symmetry}

If the magnet is flipped across the vertical axis, the north
and south poles remain just where they were, and the field
is unchanged. That means the entire magnetic field is also
unchanged, and the field at a point such as point b, along
the line of symmetry, must therefore point straight up.

If the magnet is flipped across the horizontal axis, then
the north and south poles are swapped, and the field
everywhere has to reverse its direction. Thus, the field at
points along this axis, e.g., point a, must point straight up or down.

Line up your magnet so it is pointing east-west. Choose one
of the two symmetry axes of your magnet, and measure the
deflection of the compass at two points along that axis,
as shown in the second figure, at the end of the lab.
As part of your prelab, you will use vector addition to find an equation
for $B_m/B_e$, the magnet's field in units of the Earth's, in terms
of the deflection angle $\theta$. For your first point, find the
distance $r$ at which the deflection is 70 degrees; this angle is choses because
it's about as big as it can be without giving very poor relative precision 
in the determination of the magnetic field. For your second data-point,
use twice that distance. By what factor does the field decrease
when you double $r$?

Note that the measurements are very sensitive to the
relative position and orientation of the bar magnet and
compass. You can position them accurately by laying them
both on top of a piece of graph paper, but before you set all that up,
get a preliminary estimate of the distances you'll be using, because otherwise
you can end up wasting your time.

Based on your two data-points, form a hypothesis about the variation
of the dipole's field with distance according to a power law $B\propto r^p$.
(If you've done homework problems 11 and 16 in chapter 10 of Simple Nature,
then you know what $p$ should be for an \emph{electric} dipole, based on 
vector addition of the electric fields of two charges.)

\labpart{Variation of Field With Distance: Hall Effect Magnetometer}

In this part of the lab, you will test your hypothesis about the power
law relationship $B\propto r^p$; you will find out whether the field really
does obey such a law, and if it does, you will determine $p$ accurately.

This part of the lab uses a device called a Hall effect magnetometer for
measuring magnetic fields. You don't know enough about magnetism yet to 
understand the theory behind the operation of the device, so you can
just think of it as a mysterious little probe, like a wand, that you can place at some point
in space and measure the magnetic field. The probe only measures the component of the
magnetic field vector that is parallel to its own axis. Plug the probe into the
LabPro interface, connect the interface to the computer's USB port, and
plug the interface's DC power supply in to it. Start up version 3 of Logger Pro, and
it will automatically recognize the probe and start displaying magnetic fields
on the screen, in units of mT (millitesla). The probe has two ranges, one that can
read fields up to 0.3 mT, and one that goes up to 6.4 mT. You can select either one using the
switch on the probe. To test your hypothesis with good precision, you need to obtain
data over the widest possible range of fields. Always use the more sensitive 0.3 mT scale
whenever possible, because it will give better precision for low fields. Be careful,
however, because if you expose the probe to a field that's beyond its maximum range, it
will give incorrect readings. Although you have an expectation about the direction of the
field (based both on symmetry arguments and on your qualitative results from part A),
it's a good idea to try orienting the probe along different axes to see what happens.
In general, if you want to use the probe to measure a field whose direction and magnitude
are both unknown, you need to orient the probe along two different axes, and determine the
two components separately.

Two extra complications are that the Earth's field is adding on to the magnet's field, and the
absolute calibration of the probe is very poor by default.  
You can make the computer take care of both of these issues automatically, by zeroing the
sensor (Experiment$>$Zero) when it is exposed only to the Earth's field, and aligned perpendicular to it. This causes the
computer to impose a calibration such that the Earth's field is considered to be exactly zero. You may need
to redo the calibration each time you switch scales. If you then carry out the whole measurement with the
probe and the magnet's field both aligned east-west, the Earth's field has no effect.

\labpart{Variation of Field With Angle: Hall Effect Magnetometer}

Homework problems 11 and 16 in chapter 10 of Simple Nature, predict that for
an \emph{electric} dipole, the field in the midplane is exactly half as strong
as the on-axis field, at the same distance. Test this prediction.

Also, find the magnitude of the field at an angle of 45 degrees between the midplane
and the axis. Since you don't know the direction of the field at this location based on
symmetry arguments (and you only know it very roughly based on mapping with a compass in
part A), you'll need to measure both of the field's components at this location.

As you plan your observations in this part, you'll need to think about what is the
best distance at which to place the probe. If the distance is too large, you may find that
the field is too weak to measure with good precision. If the distance is too small, then the
physical size of the probe becomes an issue, since the exact location at which the probe
measures the field is ill-defined. (The probe measures a voltage created by the field in
a sample of some material, and that sample has a finite size.)

In part C, all the fields were along a single line, and there were no angles involved.
That made it simple to get rid of the effect the Earth's field. That doesn't work in this part,
however. One way of handling the difficulty is to flip the magnet by 180 degrees, and find
the difference between the readings for the two opposite orientations of the magnet, which should
equal twice the magnet's field. The Earth's field cancels out. This means that you need a total
of four different measurements at each point in space, covering all four possible combinations of
the orientation of the probe along x or y with both orientations of the magnet.

\prelab

\prelabquestion  Suppose that
when the compass is 11.0 cm from the magnet, it is
45 degrees away from north. What is
the strength of the bar magnet's field at this location in space,
in units of the Earth's field?

\prelabquestion Find $B_m/B_e$ in terms of the deflection angle $\theta$. As a special
case, you should be able to recover your answer to P1.

\analysis

Determine the magnetic
field of the bar magnet as a function of distance. No error analysis is required. 
Look for a power-law relationship using the log-log graphing technique described in 
appendix \ref{appendix:powerlaws}. Does the power law hold for
all the distances you investigated, or only at large distances?
Compare this power law result with the result
for the variation of an \emph{electric} dipole's field with
distance.
