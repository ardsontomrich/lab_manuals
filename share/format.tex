\myappendix{format}{Format of Lab Writeups}

	Lab reports must be three pages or less, not counting your
raw data. The format should be as follows: 
\label{format-really-here}

\textbf{Title}

\textbf{Raw data} --- \emph{Keep actual observations separate from what you
later did with them.}\\
These are the results of the measurements you take down
during the lab, hence they come first. You should clearly
mark the beginning and end of your raw data, so I don't have
to sort through many pages to find your actual presentation
of your work, below. Write your raw data directly in your
lab book; don't write them on scratch paper and recopy them
later. Don't use pencil. The point is to separate facts from
opinions, observations from inferences.

\textbf{Procedure} --- \emph{Did you have to create your own methods for
getting some of the raw data?}\\
Do not copy down the procedure from the manual. In this
section, you only need to explain any methods you had to
come up with on your own, or cases where the methods
suggested in the handout didn't work and you had to do
something different. Do not discuss how you did your
calculations here, just how you got your raw data.

\textbf{Abstract} --- \emph{What did you find out? Why is it important?}\\
The ``abstract'' of a scientific paper is a \emph{short} paragraph
at the top that summarizes the experiment's results in a few
sentences. 

Many of our labs are comparisons of theory and experiment. The abstract
for such a lab needs to say whether you think the experiment was consistent
with theory, or not consistent with theory.
If your results deviated from the ideal equations, don't be afraid
to say so.
After all, this is real life, and many of the equations we
learn are only approximations, or are only valid in certain
circumstances. However, (1) if you simply mess up, it is
your responsibility to realize it in lab and do it again,
right; (2) you will never get exact agreement with theory,
because measurements are not perfectly exact --- the
important issue is whether your results agree with theory to
roughly within the error bars.

The abstract is not a statement of what you hoped to find out. It's
a statement of what you \emph{did} find out. It's like the brief statement
at the beginning of a debate: ``The U.S. should have free trade with China.''
It's not this: ``In this debate, we will discuss whether the U.S. should have
free trade with China.''

If this is a lab that has just one important numerical result (or
maybe two or three of them), put them
in your abstract, with error
bars where appropriate. There should normally be no more
than two to four numbers here. Do not recapitulate your raw
data here --- this is for your final results.

If you're presenting a final result with error bars, make sure that
the number of significant figures is consistent with your error bars.
For example, if you write a result as $323.54\pm6$ m/s, that's wrong.
Your error bars say that you could be off by 6 in the ones' place, so the
5 in the tenths' place and the four in the hundredths' place are
completely meaningless.

If you're presenting a number in scientific notation, with error bars,
don't do it like this
\begin{equation*}
  1.234 \times 10^{-89}\ \munit/\sunit \pm 3 \times 10^{-92}\ \munit/\sunit \qquad ,
\end{equation*}
do it like this
\begin{equation*}
  (1.234 \pm 0.003)\times 10^{-89}\ \munit/\sunit \qquad ,
\end{equation*}
so that we can see easily which digit of the result the error bars apply
to.

\textbf{Calculations and Reasoning}  --- \emph{Convince me of
what you claimed in your abstract.}\\
Convince me that the statements you made
about your results in the abstract follow logically from
your data. This will typically involve both calculations and
logical arguments. Continuing the debate meta\-phor, if your abstract
said the U.S. should have free trade with China, this is the rest
of the debate, where you convince me, based on data and logic,
that we should have free trade.

In your calculations, the more clearly you
show what you did, the easier it is for me to give you
partial credit if there is something wrong with your final
result. If you have a long series of similar calculations,
you may just show one as a sample. If your prelab involved
deriving equations that you will need, repeat them here
without the derivation. Try to lay out complicated
calculations in a logical way, going straight down the page
and using indentation to make it easy to understand. When
doing algebra, try to keep everything in symbolic form until
the very end, when you will plug in numbers.




\pagebreak[4]

\section*{Model Lab Writeup}

\subsection*{Comparison of Heavy and Light Falling Objects - Galileo Galilei}

\mysubsubsection{Raw Data}
(Galileo's original, somewhat messy notes go here.)

He does not recopy the raw data to make them look nicer, or
mix calculations with raw data.

\mysubsubsection{Procedure}
       We followed the procedure in the lab manual with one addition.
To make sure both objects fell at
the same time, we put them side by side on a board and then
tipped the board.

\mysubsubsection{Abstract}
Our experiment has disproved Aristotle's theory that heavy objects
fall faster than light ones.
We dropped a cannon ball
and a musket ball simultaneously from
the same height. Both hit the ground at nearly the same
time. The amount by which the cannon ball was ahead at the bottom
was observed to be $0\pm 10$ cm. This is not consistent with Aristotle,
since the error bars are not anywhere near big enough to be compatible
with his prediction that the cannonball would fall hundreds of times
faster than the musket ball due to its greater weight.

\mysubsubsection{Calculations and Reasoning}
To get a clearcut test of Aristotle's theory, we used objects
differing greatly in weight:
a cannon ball weighing two hundred pounds
and a musket ball weighing half a pound

	From a point 10 m away from the base of the tower,
the top was at a 63\degunit angle above horizontal. The
height of the tower was therefore

\begin{equation*}
		10\ \zu{m} \times \tan  63\degunit  =  20\ \zu{m}  .  
\end{equation*}
We estimated the accuracy of the 10 m horizontal
measurement to be $\pm 1$ m, with random errors mainly from
the potholes in the street, which made it difficult to lay
the meter-stick flat. If it was 11 cubits instead of 10,
our result for the height of the tower would have been 22
cubits, so our error bars on the height are $\pm 2$ cubits.

       It is common knowledge that a feather falls more
slowly than a stone, but our experiment shows that heavy
objects do not always fall much more rapidly. We do not have
any data on feathers, but we suggest that extremely light
objects like feathers are strongly affected by air
resistance, which would be nearly negligible for a
cannonball. The
Aristotelian theory is clearly wrong, since it predicts that
the cannon ball, which was 400 times heavier, would have
taken one 400th the time to hit the ground. If that was true,
then at the time the cannon ball hit the ground, the musket
ball would only have covered 1/400 of the 20-meter height
of the tower, leaving it still more than 19 m above the
ground. Our actual result of $0\pm 0.1$ m differs from
19 m by 190 standard deviations, which is extremely
improbable.
