\lab{Resonance (short version for physics 222)}

This is a simplified version of lab \ref{resonancelab}, meant to introduce
some concepts related to mechanical resonance, without any
detailed data-taking. The idea is to reinforce the relevant concepts
from physics 221 so that they can be used as a metaphor for electrical
resonances in 222.

\apparatus
\equipn{vibrator}{1/group}
\equipn{Thornton power supply}{1/group}
\equipn{stopwatch}{1/group}
\equipn{multimeter}{1/group}
\equip{banana plug cables}

\begin{goals}

\item[] Observe the phenomenon of resonance.

\item[] Learn how to visualize phases and amplitudes in a plane.
\end{goals}

\introduction

To break a wine glass, an opera singer has to sing the right
note. To hear a radio signal, you have to be tuned to the
right frequency. These are examples of the phenomenon of
resonance: a vibrating system will respond most strongly to
a force that varies with a particular frequency.

\figcaption{vw-res-mechanical}{Simplified mechanical drawing of the
vibrator, front view.}
\figcaption{vw-res-electrical}{Electrical setup, top view.}

\apparatus
In this lab you will investigate the phenomenon of resonance
using the apparatus shown in the figure. If the motor is
stopped so that the arms are locked in place, the metal disk
can still swing clockwise and counterclockwise because it is
attached to the upright rod with a flexible spiral spring. A
push on the disk will result in vibrations that persist for
quite a while before the internal friction in the spring
reduces their amplitude to an imperceptible level. This
would be an example of a free vibration, in which energy is
steadily lost in the form of heat, but no external force
\equip{pumps in energy to replace it.}
Suppose instead that you initially stop the disk, but then
turn on the electric motor. There is no rigid mechanical
link to the disk, since the motor and disk are only
connected through the very flexible spiral spring. But the
motor will gently tighten and loosen the spring, resulting
in the gradual building up of a vibration in the disk.

\observations

\labpart{ Period of Free Vibrations}

Start without any of the electrical stuff hooked up. Twist
the disk to one side, release it, and use the stopwatch to
determine its natural period of vibration. (Both here and at
points later in the lab, you can improve your accuracy by
timing ten periods and dividing the result by ten.) 

\labpart{ Damping}

Note the coils of wire at the bottom of the disk. These are
electromagnets. Their purpose is not to attract the disk
magnetically (in fact the disk is made of a nonmagnetic
metal) but rather to increase the amount of damping in the
system. Whenever a metal is moved through a magnetic field,
the electrons in the metal are made to swirl around. As they
eddy like this, they undergo random collisions with atoms,
causing the atoms to vibrate. Vibration of atoms is heat, so
where did this heat energy come from ultimately? In our
system, the only source of energy is the energy of the
vibrating disk. The net effect is thus to suck energy out of
the vibration and convert it into heat. Although this
magnetic and electrical effect is entirely different from
mechanical friction, the result is the same. Creating
damping in this manner has the advantage that it can be made
stronger or weaker simply by increasing or decreasing the
strength of the magnetic field.

Turn off all the electrical equipment and leave it
unplugged. Connect the circuit shown in the top left of the
electrical diagram, consisting of a power supply to run the
electromagnet plus a meter . You do not yet need the power
supply for driving the motor. The meter will tell you how
much electrical current is flowing through the electromagnet,
which will give you a numerical measure of how strong your
damping is. It reads out in units of amperes (A), the metric
unit of electrical current. Although this does not directly
tell you the amount of damping force in units of newtons
(the force depends on velocity), the force is proportional to the current.

Once you have everything hooked up, check with your
instructor before plugging things in and turning them on. If
you do the setup wrong, you could blow a fuse, which is no
big deal, but a more serious goof would be to put too much
current through the electromagnet, which could burn it up,
permanently ruining it. Once your instructor has checked
this part of the electrical setup she/he will show you how
to monitor the current on the meter to make sure that you
never have too much.

The $Q$ of an oscillator is defined as the number of
oscillations required for damping to reduce the energy of
the vibrations by a factor of 535 (a definition originating
from the quantity $e^{2\pi}$). As planned in your prelab,
measure the $Q$ of the system with the electromagnet turned
off, then with a current of 0.25 A through the electromagnet,
and then 0.50 A. You will be using these two current
values throughout the lab.

\labpart{ Frequency of Driven Vibration}

Now connect the lab's DC power supply to the terminals on the
motor labeled ``motorpanschlu$\beta $.'' 
The coarse and fine adjustments to the speed of the
motor are marked ``gro$\beta $'' (gross) and ``fein'' (fine).

Set the damping current to the higher of the two values.
Turn on the motor and drive the system at a frequency very
different from its natural frequency. You will notice that
it takes a certain amount of time, perhaps a minute or two,
for the system to settle into a steady pattern of vibration.
This is called the steady-state response to the driving force of the motor.

Does the system respond by vibrating at its natural
frequency, at the same frequency as the motor, or at some
frequency in between?

\labpart{ Resonance}

With your damping current still set to the higher value, try
different motor frequencies, and observe how strong the
steady-state response is. At what motor frequency do you
obtain the strongest response?

\labpart{ Resonance Strength}

Set the motor to the resonant frequency, i.e., the frequency
at which you have found you obtain the strongest response.
Now measure the amplitude of the vibrations you obtain with
each of the two damping currents. How does the strength of
the resonance depend on damping?

\labpart{ Phase Response}

If the disk and the vertical arm were connected rigidly,
rather than through a spring, then they would always be in
phase. For instance, the disk would reach its most extreme
clockwise angle at the same moment when the vertical arm was
also all the way clockwise. But since the connection is
\emph{not} rigid, this need not be the case. Find a
frequency significantly below the resonant frequency, at
which the amplitude of the steady-state response is perhaps
one tenth of the value it would have at resonance. What do
you observe about the relative phase of the disk and the
vertical arm? Are they in phase or out of phase? You can
describe the phase by assigning positive phase angles to
oscillations in which the disk is ahead of the arm, and
negative phases when the disk is behind. These phase angles
can range from -180\degunit to 180\degunit. Actually
+180\degunit and -180\degunit would represent the same
thing: the oscillations have phases that are exactly the
opposite. Try to estimate roughly what the phase angle is.
You don't have any way to measure it accurately, but you
should be able to estimate it to the nearest multiple of
45\degunit. Measure the amplitude of the steady-state response as well.

Now measure the phase and amplitude of the response when the
driving force is at the resonant frequency.

Finally, do the same measurements when the driving force is
significantly above resonance.

\analysis

The point of this is to connect the mechanical analog to what
you know about the phase response of a resonant LRC circuit.
You're measuring the phase between $F$ and $x$, which is
analogous to the phase between $V$ and $q$ in electrical
terms. However, most people think of AC circuits in terms of
$V$ and $I$, not $V$ and $q$. The phase relationships you're
expecting, therefore, are those that would hold between
$F$ and $v=dx/dt$, which differ by 90 degrees from the
$F-x$ phases you actually measured as raw data.

To complete the electrical analogy, we would really prefer to
discuss the mechanical analog of impedance. The (constant) driving force from
the motor plays the role of the voltage, while the frequency-dependent
amplitude of the vibration plays the role of the current. Dividing these
two quantities gives us something analogous to impedance, and since
the driving force is always  the same, we can say that the inverse of
the amplitude is essentially a measure of the impedance.

To summarize, you have a complex impedance whose amplitude and
phase angle you can determine from your data. Plot the impedances
at the various frequencies in the complex plane.

\prelab

\prelabquestion  Plan how you will determine the $Q$ of your oscillator
in part B. [Hint: Note that the energy of a vibration is
proportional to the square of the amplitude.]
