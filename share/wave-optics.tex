\lab{Wave Optics}\label{lab:wave-optics}

\apparatus
\equip{helium-neon laser}{1/group}
\equipn{optical bench with posts \& holders   }{1/group}
\equipn{high-precision double slits}{1/group}
\equip{rulers}
\equip{meter sticks}
\equip{tape measures}
\equip{butcher paper}
\equip{black cloths for covering light sources}

\begin{goals}


\item[] Observe evidence for the wave nature of light.

\item[] Determine the wavelength of the red light emitted by your laser, by measuring a double-slit
diffraction pattern. (The part of the spectrum that appears red to the human eye covers quite a large
range of wavelengths. A given type of laser, e.g., He-Ne or solid-state, will produce one very specific
wavelength.)

\item[] Determine the approximate diameter of a human hair, using
its diffraction pattern.
\end{goals}

\widefigcaption{op-wav-fringes}{A double-slit diffraction pattern.}\label{fig:fringes}

\introduction

Isaac Newton's epitaph, written by Alexander Pope, reads:

   Nature and Nature's laws lay hid in night.

   God said let Newton be, and all was light.

Notwithstanding Newton's stature as the greatest physical
scientist who ever lived, it's a little ironic that Pope
chose light as a metaphor, because it was in the study of
light that Newton made some of his worst mistakes. Newton
was a firm believer in the dogma, then unsupported by
observation, that matter was composed of atoms, and it
seemed logical to him that light as well should be composed
of tiny particles, or ``corpuscles.'' His opinions on the
subject were so strong that he influenced generations of his
successors to discount the arguments of Huygens and Grimaldi
for the wave nature of light. It was not until 150 years
later that Thomas Young demonstrated conclusively that light was a wave.

Young's experiment was incredibly simple, and could probably
have been done in ancient times if some savvy Greek or
Chinese philosopher had only thought of it. He simply let
sunlight through a pinhole in a window shade, forming what
we would now call a coherent beam of light (that is, a beam
consisting of plane waves marching in step). Then he held a
thin card edge-on to the beam, observed a diffraction
pattern on a wall, and correctly inferred the wave nature
and wavelength of light. Since Roemer had already measured
the speed of light, Young was also able to determine the
frequency of oscillation of the light.

Today, with the advent of the laser, the production of a
bright and coherent beam of light has become as simple as
flipping a switch, and the wave nature of light can be
demonstrated very easily. In this lab, you will carry out
observations similar to Young's, but with the benefit of
hindsight and modern equipment.

\observations



\labpart{ Determination of the wavelength of red light}

Set up your laser on your optical bench. You will want as
much space as possible between the laser and the wall, in
order to let the diffraction pattern spread out as much as
possible and reveal its fine details.

Tear off two small scraps of paper with straight edges. Hold
them close together so they form a single slit. Hold this
improvised single-slit grating in the laser beam and try to
get a single-slit diffraction pattern. You may have to play
around with different widths for the slit. No quantitative
data are required. This is just to familiarize you with
single-slit diffraction.

Make a diffraction pattern with the double-slit grating. See
what happens when you hold it in your hand and rotate it
around the axis of the beam.

The diffraction pattern of the double-slit grating consists
of a rapidly varying pattern of bright and dark bars, with a
more slowly varying pattern superimposed on top. (See the figure
two pages after this page.) The rapidly varying pattern is the one that is
numerically related to the wavelength, $\lambda $, and the
distance between the slits, $d$, by the equation
\begin{equation*}
         \Delta\theta =  \lambda /d  ,  
\end{equation*}
where $\theta $ is measured in radians. To make sure you can
see the fine spacing, put your slits several meters away
from the wall. This will necessitate shining it across the
space between lab tables. To make it less likely that
someone will walk through the beam and get the beam in their
eye, put some of the small desks under the beam.
The slit patterns we're using actually have three sets of
slits, with the following dimensions:\\
\begin{tabular}{lll}
    & $w$ (mm) & $d$ (mm) \\
A   & .12 & .6 \\
B   & .24 & .6 \\
C   & .24 & 1.2 
\end{tabular}\\
The small value of $d$ is typically better, for two reasons:
(1) it produces a wider diffraction pattern, which is easier
to see; (2) it's easy to get the beam of the laser to cover
both slits. 

If your diffraction pattern doesn't look like the
one in the figure on the following page, typically the
reason is that you're only covering one slit with the beam
(in which case you get a single-slit diffraction pattern),
or you're not illuminating the two slits equally (giving
a funny-looking pattern with little dog-bones and things in it).

As shown in the figure below, it is also possible to have the
beam illuminate only \emph{part} of each slit, so that the slits
act effectively as if they had a smaller value of $d$. The beam
spreads as it comes out of the laser, so you can avoid this
problem by putting it fairly far away from the laser (at the far
end of the optical bench).

\fig{op-wav-hit-slits}

Think about the best way to measure the spacing of the
pattern accurately. Is it best to measure from a bright part
to another bright part, or from dark to dark? Is it best to
measure a single spacing, or take several spacings and
divide by the number to find what one spacing is? Do it.

Determine the wavelength of the light, in units of
nanometers. Make sure it is in the right range for red
light.

Check that the $\Delta\theta$ you obtain is in the range
predicted in prelab question P1. In the past, I've seen
cases where groups got goofy data, and I suspect that it
was because they were hitting a place on the slits where
there was a scratch, bump, or speck of dust.

\labpart{ Diameter of a human hair}

Pull out one of your own hairs, hold it in the laser beam,
and observe a diffraction pattern. It turns out that the
diffraction pattern caused by a narrow obstruction, such as
your hair, has the same spacing as the pattern that would be
created by a single slit whose width was the same as the
diameter of your hair. (This is an example of a general
theorem called Babinet's principle.) Measure the spacing of
the diffraction pattern. (Since the hair's diameter is the
only dimension involved, there is only one diffraction
pattern with one spacing, not superimposed fine and coarse
patterns as in part A.) Determine the diameter of your hair.
Make sure the value you get is reasonable, and compare with
the order-of-magnitude guess you made in your prelab writeup.

\prelab

Read the safety checklist.

\prelabquestion  Look up the range of wavelengths that the human eye perceives as red.
With $d=0.6$ mm, predict the lowest and highest possible values of $\Delta\theta$
that could occur with red light.

\prelabquestion  It is not practical to measure $\Delta\theta$ directly using a
protractor. Suppose that a lab group finds that 27 fringes extend over 29.7 cm
on their butcher paper, which is on a wall 389 cm away from the slits. They
calculate $\Delta\theta=\tan^{-1}(29.7\ \zu{cm}/(27\times389\ \zu{cm}))$
$=2.83\times10^{-3}\ \zu{rad}$.
Simplify this calculation using a small-angle approximation, and show that the resulting
error is negligible.

\prelabquestion  Make a rough order-of-magnitude guess of the diameter of a human hair.

\analysis

Determine the wavelength of the light and the diameter of
the hair, with error bars.
