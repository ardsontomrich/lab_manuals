\lab{The Moment of Inertia}\label{lab:moment-of-inertia}

\apparatus
\equipn{meter stick with hardware attached}{1/group}
\equipn{nail}{1/group}
\equipn{fulcrum}{1/group}
\equip{200 g slotted masses (old, gray ones)}
\equip{hooked masses  (in lab benches in 415)}
\equip{duct tape}
\equipn{computer with LabPro interface}{1/group}
\equipn{LabPro-compatible photogate}{1/group}
\equipn{stand and right-angle clamp for photogate}{1/group}
\equip{Vernier calipers}
\equip{stopwatches}

\goal{Test the equation $\tau_{total}=I\alpha $, which relates an
object's angular acceleration to its own moment of inertia
and to the total torque applied to it.}

\introduction

Newton's first law, which states that motion in a straight
line goes on forever in the absence of a force, was
especially difficult for scientists to work out because
long-lasting circular motion seemed much more prevalent in
the universe than long-lasting linear motion.  The sun, moon
and stars appeared to move in never-ending circular paths
around the sky.  A spinning top could continue its motion
for a much longer time than a book sliding across a table. 
Ren\'{e} Descartes (b. 1596) came close to stating a law of
inertia like Newton's, but he thought that matter was made
out of tiny spinning vortices, like whirlwinds of dust. 
Galileo, who among Newton's predecessors came closest to
stating a law of inertia, was also confused by the issue of
circular versus linear inertia.  An advocate of the
Copernican system, in which the apparent rotation of the
sun, stars and moon was due to the Earth's rotation, he knew
that the apparently motionless ground, trees, and mountains
around him must be moving in circles as the Earth turned. 
Was this because inertia naturally caused things to move in circles?

Newton, like other giants of science, saw how to focus on
the simple rather than the complex.  His law of inertia was
completely linear.  In his view, all the common examples of
circular motion really involved a force, which kept things
from going straight.  In the case of a spinning top, for
instance, Newton (a confirmed atomist) would have visualized
an atoms in the top as being acted on by some kind of sticky
force from the other atoms, which would keep it from flying
off straight.  Linear motion was the simplest type, needing
no forces to keep it going.  Circular motion was more
complex, requiring a force to bend the atoms' trajectories into circles.

Even though circular motion is inherently more complicated
than linear motion, some very close analogies can be made
between the two in the case where an object is spinning
rigidly.  (An examples of nonrigid rotation would be a
hurricane, in which the inner parts complete a rotation more
rapidly than the outer parts.)  In analogy to Newton's first
law, $F_{total}=ma$, we have
\begin{equation*}
	\tau_{total}=I\alpha    \qquad   ,
\end{equation*}
where the angular acceleration $\alpha $ replaces the linear
acceleration $a$, the total torque plays the role given to
the total force, and the moment of inertia $I$ is used
instead of the mass. In this lab, you are going to release
an unbalanced rotating system --- a meter stick on an axle
with weights attached to it --- and measure its angular
acceleration in response to the nonzero gravitational torque on it.

   Every equation you learned for constant acceleration can
also be adapted to the case of rotation with constant
angular acceleration, simply by translating all the
variables.  For instance, the equation $v_f^2=2ax$
for an object accelerating from rest can be translated into
the valid rotational formula $\omega_f^2=2\alpha\theta$.

The moment of inertia is defined as $I=\sum mr^2$,  where
$m$ can be thought of as the mass of an individual atom
comprising the rotating body, and $r$ is the distance of
that atom from the axis of rotation.

The word ``moment'' in ``moment of inertia'' does not refer
to a moment in time, but is used instead in a more
old-fashioned sense of ``importance'' or ``weight,'' as in
``matters of great moment.''  The idea is that the factor of
$r^2$ gives more importance to the an atom that is far from
the axis of rotation.  Because the symbol $I$ is used, there
is a tendency for students to refer to it as ``inertia,''
but inertia is a different and nonquantitative concept,
referring to the tendency of objects to stay at rest or stay in motion.

In practice, it is not practical to carry out a sum over all
the atoms.  The object whose rotation you will study in this
lab will consist of a meter stick pivoting at its center,
with various weights hanging from it in various places. 
Both the hanging weights and the meter stick itself will
contribute to the moment of inertia.  To a good approximation,
each hanging weight can be treated as if all its atoms were
concentrated at its center.  Calculus can also be used to
derive formulae for the moments of inertia of objects of
various shapes, such as a sphere, a cylinder rotating along
its axis, etc.  One such formula is $I=(1/12)ML^2$ for the moment of
inertia of a rigid rod rotating about an axis passing
perpendicularly through its center.  You can use this
as a good approximation for the meter stick's
contribution to the moment of inertia, with $L=1$ m.

\fig{me-moi-balancing}

\mysubsubsection{Preliminaries}

The meter stick is supported on the fulcrum via a nail
through the hole in its center. You want to start by
producing a balanced arrangement of weights attached to the
meter stick, as in figure (a) above. The idea is that if you
first balance this configuration carefully, then you know
that the net gravitational torque on it is zero. If you then
hang another weight from the previously empty hanger, as in
(b), then you know that the total torque simply equals the
torque produced by the earth's gravitational force
on the added weight.

\fig{me-moi-stability}

To balance your initial configuration, you need to attach the
slotted weight to the meter stick and then have a way to slide
it around. Wrap a tight sleeve of duct
tape, sticky side out, so that it can slide back and forth.
Then stick the slotted weight to it. As described below, you will
adjust the center of mass both vertically and horizontally to coincide
with the nail. To get reasonably close vertically, make sure to put the
slotted weight near the center of the stick vertically; otherwise you
won't have enough range of adjustment available later.

Once have the slotted weight close to the right horizontal position,
use tape to lock the weight securely in place, and finish up by
doing a final adjustment using the sliding bracket.

The left-right adjustments should not take more than about 15 minutes.
If they take longer than that, you're doing something wrong. It's not
even important to get them more than approximately right, because
at the end you'll be determining the moment of inertia from the graph
of $\tau$ versus $\alpha$, and any maladjustment will only have an effect on the y-intercept of
this graph, not its slope.

In addition to the left-right adjustments, what is less obvious that it
makes a difference how high the weights are placed. That is,
the center of mass of the whole balanced setup must coincide
both vertically and horizontally with the nail. This is the reason
for using bolts as two of the weights; they can be easily moved up and down.

The concept
is shown in the figure above using a rectangle in place of
the actual apparatus.
In (d), there will always be a
clockwise torque on the rectangle, because the center of
mass is to the right of the nail.
In (e), there is zero torque if the rectangle is initially
released from this horizontal position, but the equilibrium
is unstable, because the center of mass is above the axis of
rotation. Our experiment depends on the cancellation of the
gravitational torques on everything but the extra weight,
but in a case like (e), this assumption would only be valid
when the apparatus was initially released from horizontal.
Lat\-er in the motion, there would be an undesired and unknown
extra torque. Although it is visually obvious in this figure
that the rectangle's center of mass is too high, you can't
tell visually with the actual apparatus. The way to tell if
the center of mass is too high is that if you tilt the meter
stick a little bit to the right, it immediately accelerates
clockwise, whereas if you tilt it a little to the left, it
accelerates counterclockwise.

In (f), we have a stable equilibrium. Again, there is an
unknown, undesired torque unless the rectangle just happens
to be horizontal. You can tell if you have this situation
because the apparatus can swing back and forth about its
stable equilibrium position.

You want a \emph{neutral} equilibrium, i.e., no matter what
angle you release it from, the meter stick just stays there.
A good way to home in on a neutral equilibrium is to first obtain
a stable equilibrium, and use a stopwatch to time the oscillations
around equilibrium. If you then raise one of the bolts, getting
closer to a neutral equilibrium, the period of the oscillations
will increase. Theoretically this period becomes infinite at neutral
equilibrium, but in reality what happens is that if you're close
to neutral equilibrium, static friction will keep the stick still,
no matter what angle you release it at.

Some of the hardware is permanently attached, so you can't take it
off and weigh it separately. The mass of each meter stick is written
on its back. The other masses are as follows:\\
\begin{tabular}{ll}
3/4-inch bolt       & 204.4 g \\
3/4-inch nut        &  49.7 g \\
small hose clamp (around ruler) & 17.2 g \\
medium hose clamp & 19.4 g \\
large hose clamp & 20.1 g\\
sliding bracket and hook & 15.7 g \\
\end{tabular}\\
To tell whether the hose clamp wrapped around the 3/4-inch nut
is medium or large, look for the X marked on the tightening machines
of the large ones. (The large ones also have about an inch and a half of
extra steel band sticking out.)

\observations

Now add the extra weight so that the meter stick is slightly
unbalanced.  The idea of this lab is to release the meter
stick and use the photogate to find how quickly it is moving
once it has rotated through some angle, using the photogate
to find the amount of time required for the tip of the meter
stick to pass through the photogate.  From your measurement
of $\Delta t$ using the photogate, you can find $\omega=\Delta \theta/\Delta t$, 
which is an approximation to the meter
stick's final angular velocity.

Once you know the meter stick's final value of $\omega $,
you can extract experimental moment of inertia.

This lab is subject to systematic errors due to friction and to
imperfect left-right adjustment of the center of mass.
Both of these cause a constant to be added to the torque.
To eliminate this error, you should take data with multiple
values of the extra hanging weight, and extract the experimental
moment of inertia from the slope of a graph of $\tau$ as a function
of $\alpha$. Any constant added onto the torques merely changes
the y-intercept of this graph, without changing its slope.

Tips:

\begin{itemize}
\item[] You may want to put something under the fulcrum base to
raise everything up higher.

\item[] Although the balanced configuration, with $\tau_{total}=0$,
still has $\tau_{total}=0$ no matter what angle it is at,
the torque exerted by the extra weight does depend a little
on what angle the meter stick is at.  This is because of the
factor of $\sin\theta_{rF} $ in the definition $\tau=rF\sin\theta_{rF} $. 
(Note that $\theta_{rF}$ is not the same as the angle $\theta$
shown in the figure below.)
Since the torque is not constant, the
angular acceleration is not constant, leading to complications.
 You can avoid this problem by confining all your measurements
to a fairly small range of positions near horizontal.  As
long as $\theta_{rF} $ is fairly close to $90\degunit$, $\sin\theta_{rF} $ 
is extremely close to 1, and it is a good enough
approximation to assume a constant torque $rF$
producing a constant angular acceleration. For instance, as
long as $\theta_{rF} $ is within $20\degunit$ above or below
horizontal, $\sin\theta_{rF} $ changes by no more than 0.06.

\item[] Although you want to work only with nearly horizontal
positions of the meter stick so that the torque is
approximately constant, you also need to make sure that the
total angle traversed by the meter stick is still reasonably
large compared to the angle traversed while the meter stick
is blocking the photogate.  Otherwise your measurement of
$\omega =\Delta \theta/\Delta t$ will not be a good approximation to
the final instantaneous angular velocity.

\item[] As you will find in your prelab, the angular acceleration
depends on the square of the angle $\Delta\theta$. Measuring this angle
accurately is therefore vital in order to get a good result.
\emph{A protractor cannot measure an angle this small with
sufficient accuracy.} Use trigonometry to determine this angle.

\item[] It's easiest if you use radian measure throughout.  The
equation $\tau_{total}=I\alpha $ is only true if $\alpha$ is
measured in $\zu{radians}/\sunit^2$.

\end{itemize}

\prelab

\prelabquestion  The figures below show the angles that you will measure
directly. In addition you will measure the time for which the photogate
is blocked. Derive an equation for the experimental value of the
final angular velocity, expressed in terms of quantities you
will actually measure directly. Note that this
lab is exactly analogous to lab \ref{lab:acceltwod}.

\fig{me-moi-angles}

\prelabquestion  Why would it not be meaningful to try to discuss the
meter stick's velocity, rather than its angular velocity?

\prelabquestion Find the meter stick's experimental angular acceleration
$\alpha_{exp}$ in terms of $\theta$ and $\omega_f$, and then substitute
in your answer from P1 to get $\alpha$ in terms of quantities that
you can measure directly.

\analysis

Use the slope of the graph of $\tau$ versus $\alpha$ to find the
experimental moment of inertia. See appendix \ref{appendix:graphing}
on p.~\pageref{appendix:graphing} for how to do this.

Compare with the theoretical
moment of inertia calculated by knowing how the mass is distributed.

For extra credit,
correct for the fact that the bolts have a certain length,
so they contribute more to the moment of inertia than a pointlike mass.
For this correction, you can use the parallel axis theorem, which is
stated in section 15.10 of \emph{Mechanics}. The size of
the resulting correction to the moment of inertia is on the order of a few
percent.
