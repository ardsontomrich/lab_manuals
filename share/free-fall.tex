\lab{Free Fall}\label{lab:free-fall}

\apparatus
\equip{two stations:}
\equip{Behr free-fall column and weight}
\equip{plumb bob}
\equip{spark generator (CENCO)}
\equip{paper tape}
\equip{switch for electromagnet}

\goal{Find out whether it is $\Delta v/\Delta x$ or
$\Delta v/\Delta t$ that is constant for an object
accelerating under the influence of gravity.}

\introduction

A fundamental and difficult problem in pre-New\-ton\-ian physics
was the motion of falling bodies.  Aristotle had various
incorrect but influential ideas on the subject, including
the assertions that heavier objects fell faster than lighter
ones and that the object only sped up for a short while
after it was dropped and then continued on at a constant
speed. Even among Renaissance scientists who disagreed with
Aristotle's claim that the object no longer sped up after a
while, there was a great deal of confusion about whether it
was $\Delta v/\Delta x$ or $\Delta v/\Delta t$ that was
constant.  It seems obvious to modern physicists that they
could not both be constant, but it was not at all obvious to
authorities such as Domingo de Soto and Albert of Saxony. 
Galileo started out thinking they were both constant, then
realized this was mathematically impossible, and finally
determined from experiments which one was actually constant.

The main reason why the confusion persisted for two thousand
years was that the methods for measuring time were
inaccurate, and the time required for an object to fall was
very short.  Galileo was able to make settle the issue
because he figured out how to use a pendulum to measure time
accurately, and also came up with the idea of effectively
slowing down the motion by studying objects rolling down an
inclined plane, rather than objects falling vertically.  He
then found how to extrapolate from the case of an object
rolling down an inclined plane at an angle $\theta $ to the
ideal case of $\theta $=90\degunit, which would be the same
as free fall.  Galileo's task would have been a lot simpler
if he'd had accurate enough devices for measuring time,
because then he could have simply carried out measurements
for objects falling vertically.  That's what you'll do today.

\fig{me-fre-behr}

\mysubsection{Setup}

The apparatus consists of a 2-meter tall column with a paper
tape running down it.  A weight is held at the top with an
electromagnet and then released, falling right next to the
paper tape.  (An electromagnet is an artificial magnet that
works when you put an electric current through it, unlike a
permanent magnet, which does not require power.)  A spark
generator is hooked up to the two vertical wires, and as the
weight falls, sparks cross the gap from the first wire to
the metal flange on the weight, then from the flange to the
other wire.  Sparks are produced only briefly, at regular
intervals of 1/60 of a second.  On their way, the sparks go
through the paper tape, making dots on it that show the
location of the weight at 1/60-s intervals.


First, unplug the spark generator so you don't get shocked
while you're getting things ready.  Use the switch made from
a regular light switch to turn on the magnet at the top of
the column, which operates on 7 volts from the lab's DC
power circuits.  Insert the plumb bob, hanging from the
magnet.  Use the three screws on the feet of the column to
level the apparatus so the plumb bob's string is parallel to the wire.

Replace the plumb bob with the weight.  Pull fresh tape up
from the roll at the bottom, and get the tape straight and
centered on the wire.

Plug in the spark generator, and put the function knob on
``line,'' which means it will base its cycle of sparks on
the AC power from the wall, which switches directions once
every sixtieth of a second.  The red LED should light up. 
From now on, do not press the thumb switch to activate the
sparks unless you are sure nobody is near the vertical
wires.  Try it out, and see if you get a spot at the top of
the tape, where the weight currently is.

\mysubsection{Observations}

Hold down the thumb switch to make the sparks start, flip
the switch to release the weight, and wait until the weight
has fallen in the cup at the bottom before releasing the
thumb switch.  You want a nice straight line of dots on the
tape, going all the way from the top to the bottom --- you
may have to make adjustments and try a few times before
getting a good tape.  Take your tape off, and measure the
locations of the dots accurately with a two-meter stick.

Sometimes even if you carefully straight\-en the
paper tape, etc., you still won't be able to avoid having
some ``gap teeth'' on your tape. If this happens after several
tries, just use your best run, and mark in fake X's with a pencil in the relevant region.
Don't even worry about getting them in the right place. All that
will happen is that your graphs will have little glitches, which you
can ignore visually. The ony thing that's critical is that you
add the right \emph{number} of X's.

\analysis

Since the sparks start before you release the electromagnet,
the first dot at the very top of the tape will give the
starting position of the weight.

If you consider any adjacent pair of dots (avoiding the top
and bottom ones), then measuring the distance $\Delta x$ between them
allows you to calculate an approximation to the speed of the
weight, which you can think of as being its speed at the
point half-way between the two dots.

Make one plot of speed versus time and another of speed
versus distance, preferably using a computer, since you will
have about thirty data points, and it would be tedious to
plot them all by hand.

Determine whether your data are consistent with constant
$\Delta v/\Delta x$ or $\Delta v/\Delta t$ or neither. 

\selfcheck

Appendix \ref{appendix:graphing} discusses graphing.
The graphing for this lab is time-consuming without a computer; since we
have a limited number of computers in lab, you may want to go to one of the other campus
computer labs for this. Determine which quantity is constant.

\prelab

\prelabquestion Consider the quantities $x$, $\Delta x$, $t$, and $\Delta t$, which are
measured more or less directly in this lab. Which of these would have a single value
throughout the whole motion of the falling weight?

\prelabquestion Explain, based on the meaning of the symbols $\Delta$ and $/$,
why $\Delta v/\Delta x$ and $\Delta v/\Delta t$ have to be notations for \emph{numbers},
not descriptions of \emph{graphs}. How would these two numbers relate to the two
graphs?

\prelabquestion
Suppose that, once you have data, $\Delta v/\Delta x$ turns out to be constant, and
$\Delta v/\Delta t$ isn't. Draw a pair of graphs that would lead to this conclusion.
