\lab{The Momentum Vector}\label{lab:momentum-vector}

\apparatus
\equipn{LabPro-compatible photogates (in lab benches in 415)}{3/group}
\equipn{Linux computer with LabPro interface}{1/group}
\equipn{1-inch steel ball}{1/group}
\equipn{1-inch aluminum ball}{1/group}
\equip{plastic rulers}
\equip{protractor}
\equip{scotch tape}
\equip{small stands}
\equipn{short rod}{1/group}
\equipn{right-angle clamps}{6/group}
\equip{wood blocks}

(The small stands, extra rods, and clamps are necessary in order
to get the photogate low enough.)

\goal{Test whether momentum is conserved in a collision of two balls.}

\introduction

Pool players have an intuitive feeling for conservation of
momentum: they can visualize the results of a collision of
two pool balls in advance.  They also know that certain
shots are impossible.  For instance, there is no way to make
the cue ball bounce back directly from a collision with
another ball (except by putting spin on it, which creates an
external friction force with the felt).  They understand
that the angles are important, so without knowing it, they
are doing mental estimates involving momentum as a vector: a
quantity that has both magnitude and direction.

In this lab, you will be studying collisions similar to the
collision of the cue ball with an initially stationary ball.
One of the basic principles involved is the law of conservation of momentum,
which tells us that the initial momentum vector of the incoming ball
is equal to the vector sum of the momenta of the two balls after the
collision.

\fig{me-mom-vectors}

\mysubsubsection{The technique}

The idea is to set up an off-center collision, as shown
below, and measure the initial and final speeds of the balls
using the photogates and the computer.  
Use the photogate application that runs on the Linux computers.
Daisy-chain the three photogates together and plug them into
the interface.

Since all three photogates are plugged into a single port on
the interface, the computer can't tell one from the other.
Make sure that the time intervals in which they are blocked do
not overlap.

\fig{me-mom-ramp}

To reproduce the same initial speed for the projectile (ball
1), you can build a little ramp out of two plastic rulers
taped together at a 90-degree angle.  A block of wood can be
taped in the ramp at the top to keep them braced.  The block
of wood also serves as a convenient reference point: you can
release the ball from the point where it touches the block.

You should choose a completely asymmetrical setup: two balls
of different masses, and a collision in which the projectile
does not hit the target head-on.

It is critical that you position the target ball at exactly
the same place every time. Marking the table and placing the
ball on the mark is not good enough. The best technique is
to put a piece of scotch tape on the table and use a
ball-point pen to make a tiny impression in it for the
target ball to sit in.

\mysubsubsection{Tips}

You want to avoid conditions for which any of the speeds
involved are too slow, because then the balls tend to be
accelerated, decelerated, or deflected by tiny bumps in the
tabletop.  If you notice the balls wandering and wavering as
they roll, they are going too slow.  Generally speaking,
sufficiently high speeds are achieved if the ramp is at
least 7 cm high.  Using the heavier ball as the projectile
helps to keep the final speeds high.

A good way to test whether your speeds are sufficient is to
measure the angles at which the balls emerge from the
collision, and see if they are the same every time, to
within a tolerance of a few degrees.  If the angles are not
reproducible to this level of variation, then the balls are
not going fast enough.

Note that at the instant of collision, the balls are
touching, but their centers are not at the same point.  This
means you have to be careful about how you measure the angles.

Try to position the photogate so that its beam cuts through
the center of the ball, along a diameter.
The balls are equal in size, so in theory this adjustment just scales all the momenta equally without affecting conservation of momentum.
However, the relative precision of the results is improved by measuring the longer times
that result from letting the full diamater of the ball cut through the beam.
To get the photogate at the right height, connect a stand to
a horizontal rod using a right-angle bracket, and then use another
right-angle bracket to connect that to the photogate.

You will be putting the photogate in three different
positions to measure the three velocities. How far from the
collision should you place it? It should be as close as
possible to the collision, because the balls do gradually
slow down as they roll, and you want to know the speeds
immediately before and after the collision. However, the
balls bounce a little immediately after the collision, so
don't put it so close to the collision that they are
still bouncing when they go through it. They projectile
ball also bounces as it comes off of the ramp.

Some areas on some lab benches are not very level. You can
figure this out by releasing the ball and seeing if it accelerates.

\prelab

\prelabquestion  Draw an example of a collision, showing the balls before
and after it happens, in which $|\vc{p}_{1i}|=0.020\ \kgunit\unitdot\munit/\sunit$,
$|\vc{p}_{1f}|=0.010\ \kgunit\unitdot\munit/\sunit$, and
$|\vc{p}_{2f}|=0.010\ \kgunit\unitdot\munit/\sunit$, but momentum was \emph{not}
conserved. (As in the actual lab, the target ball starts at rest.) Explain.

\prelabquestion  If the magnitude of the initial momentum is the same as
the magnitude of the total final momentum, does that mean
momentum was conserved?

\selfcheck

Analyze your data without error analysis.
Check whether
momentum appeared to be at least approximately conserved.

\analysis

Test whether momentum was conserved, using vector addition.
Take into account the random errors
in your measurements.

You should have opposite signs for the components of the
balls' final momenta in the direction perpendicular to the
projectile's original direction of motion.
