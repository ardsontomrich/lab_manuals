\lab{The Photoelectric Effect}\label{lab:photoelectric-effect}

\apparatus
\equip{Hg gas discharge tube}
\equip{light aperture assembly}
\equip{lens/grating assembly}
\equip{photodiode module, support base, and coupling rod}
\equip{pieces of plywood}
\equip{green and yellow filters}

\begin{goals}

\item[] Observe evidence that light has particle properties as
well as wave properties.

\item[] Measure Planck's constant.
\end{goals}

\introduction

The photoelectric effect, a phenomenon in which light shakes
an electron loose from an object, provided the first
evidence for wave-particle duality: the idea that the basic
building blocks of light and matter show a strange mixture
of particle and wave behaviors. At the turn of the twentieth
century, physicists assumed that particle and wave phenomena
were completely distinct. Young had shown that light could
undergo interference effects such as diffraction, so it had
to be a wave. Since light was a wave composed of oscillating
electric and magnetic fields, it made sense that when light
encountered matter, it would tend to shake the electrons. It
was only to be expected that something like the photoelectric
effect could happen, with the light shaking the electrons
vigorously enough to knock them out of the atom. The best
theoretical estimates, however, were that light of ordinary
intensity would take millions of years to do the trick ---
it would take that long for the electron slowly to absorb
enough energy to escape.

The actual experimental observation of the photoelectric
effect was therefore an embarrassment. It started up
immediately, not after a million years. Albert Einstein,
better known today for the theory of relativity, was the
first to come up with the radical, and correct, explanation.
Einstein simply suggested that in the photoelectric effect,
light was behaving as a particle, now called a photon. The
beam of light could be visualized as a stream of machine-gun
bullets. The electrons would be small targets, but when a
``light bullet'' did score a hit, it packed enough of an
individual wallop to knock the electron out immediately.
Based on other experiments involving the spectrum of light
emitted by hot, glowing objects, Einstein also proposed that
each photon had an energy given by
\begin{equation*}
         E  =  hf  ,  
\end{equation*}
where $f$ is the frequency of the light and $h$ is Planck's constant.

In this lab, you will perform the classic experiment used to
test Einstein's theory. You should refer to the description
of the experiment in your textbook. Briefly, you will expose
the metal cathode of a vacuum tube to light of various
frequencies, and determine the voltage applied between the
cathode and anode that just barely suffices to cut off the
photoelectric current completely. This is known as the
stopping voltage, $V_s$. According to Einstein's theory, the
stopping voltage should obey the equation
\begin{equation*}
         e V_s  =  hf - E_s   ,  
\end{equation*}
where $E_s$ is the amount of energy
required by an electron to penetrate the surface of
the cathode and escape.

\figcaption{mo-pho-optics}{Optical setup.}

\setup

You can use the Hg gas discharge tube to produce monochromatic
light with the following wavelengths:\label{hg-wavelengths}

\begin{tabular}{ll}
color   &wavelength (nm)\\
ultraviolet &  365\\
violet   &405\\
blue   &436\\
green  & 546\\
yellow  & 578
\end{tabular}

The diffraction grating splits up the light into these
lines, so you can make one line at a time enter the
photodiode. Slit 1 slides into the slot in the front of the
discharge tube. The lens serves to create focused images of slit 1 at
the photodiode.  The lens and diffraction grating are housed
in a single unit, which is attached to a pair of rods (not
shown) projecting from slit 1. Do not drop the lens and
diffraction grating --- I have already damaged one by
dropping it, and they cost \$200 to replace. For measurements
with the green and yellow lines, green and yellow filters
are used to help eliminate stray light of other colors ---
they stick magnetically on the front of the collimator tube.
Slit 2 and the collimator tube keep stray light from getting in.

The photodiode module is held on top of a post on a rotating
arm.  The ultraviolet line is invisible, but the front of
slit 2 is coated with a material that fluoresces in UV
light, so you can see where the line is.

\figcaption{mo-pho-circuit}{Circuit.}

\mysubsubsection{Circuit}

The circuit in fig. (a) above is the one shown in textbooks
for this type of experiment. Light comes in and knocks
electrons out of the curved cathode. If the voltage is
turned off, there is no electric field, so the electrons
travel in straight lines; some will hit the anode, creating
a current referred to as the photocurrent. If the voltage is
turned on, the electric field repels the electrons from the
wire electrode, and the current is reduced or eliminated.
The stopping voltage would be measured by increasing the
voltage until no more current was flowing. We used to use a
setup very similar to this in this course, but it was
difficult to get good data because it was hard to judge
accurately when the current had reached zero.

The circuit we now use, shown in fig. (b), uses a cute trick
to determine the stopping voltage. The photocurrent
transports electrons from the cathode to the anode, so a net
positive charge builds up on the cathode, and a negative
charge on the anode.  As the charge and the
voltage increase, the photocurrent is reduced, until finally
the voltage reaches the stopping voltage, and no more
current can flow. You then read the voltage off of the
voltmeter. When you have the next color of light shining on
the cathode, you momentarily close the switch, discharging
the photodiode, and then take your next measurement. The
only disadvantage of this setup is that you cannot adjust
the voltage yourself and see how the photocurrent varies with voltage.

\setup

Move the housing containing the grating and lens until you
get a good focus at the front of the photodiode box. The
square side needs to be facing away from the discharge tube.

Diffraction patterns are supposed to be symmetric, i.e., the $m=1$
and $m=-1$ maxima should be identical. In reality, there is something
strange about this setup that can cause the shorter wavelength lines
(especially the UV line) to be extremely dim on one side. Check which
one is brighter on your apparatus.

Just because the light gets in through slit 2 does not mean
it is getting in to the photodiode. The original design of
the apparatus allowed the photodiode module to twist around
on its post, and it had to be adjusted carefully by trial and
error. Because students were getting frustrated with this, I
epoxied the photodiode modules onto their posts in the right
orientation. This makes it impossible to disassemble the
apparatus and put it in its storage box, but should get rid
of the hassles with orienting it. However, you should still
check that it's oriented correctly, because it's possible
that your setup was a little different from mine,
the epoxy can be cracked by rough handling, and the screw
at the base of the post can also get loose. There
are three things you should check to make sure the orientation
is right: (1) Sighting along the tube like a gun, you should
see that it looks like it's lined up with the center of the
grating. (2) The tube can be lifted out on a hinge so that
you can see the glass photodiode tube inside the box; check that light
is actually falling on the opening on the side of the tube.
(3) Take data using the UV line. If
you don't get a bigger voltage for this line than for the
others, then the light is not making it in to the photodiode.

\observations

You can now determine the stopping voltages corresponding to
the five different colors of light.

Hints:

\begin{itemize}
\item[] The biggest possible source of difficulty is stray light.
The room should be dark when you do your measurements.

\item[] The shortest wavelengths of light (highest frequencies),
for which the energy of the photons is the highest, readily
produce photoelectrons. The photocurrent is much weaker for
the longer wavelengths. Start with the short-wavelength
line and graduate to the more difficult, lower frequencies.
Don't forget the filters for the yellow and green lines!

\item[] If the button to zero the voltage doesn't work, it is
because the batteries are dead.

\item[] When you hit the button to zero the voltage, it may actually
pop up high and then come back down. This is normal. (It's acting
like an RC circuit with a long RC time constant).

\item[] Check the batteries in your photodiode module before
you start, using the two banana plugs designed for this purpose.
If your batteries are dead, you need to replace them. I've also
seen cases where the batteries are on the borderline at the beginning
of the lab, and then die completely during the lab; in this
situation, you'll notice that the stopping voltages you're
measuring change over the course of the lab, and don't make
sense. It won't hurt to check the batteries at the end of the
lab as well as at the beginning.

\item[] Where the lines hit the white front of slit 2, they should
be sharp, and should not overlap. You can adjust the focus
by moving the lens and grating in or out. If you can't get a
good focus, check and make sure that the square side of the
unit is away from the Hg tube.

\item[] The photodiode module can be rotated on its post so that
the light goes straight down the tube. If you don't line it
up correctly, you'll be able to tell because the voltage
will creep up slowly, rather than shooting up to a certain
value and stopping. There is a screw that is supposed to
allow you to lock the photodiode into position at the
correct angle. Make sure to loosen the screw before trying
to aim the photodiode, and lock it once it's aimed
correctly. If your photodiode won't lock in place, you need
to tighten the aluminum post that forms the base of the box.
\end{itemize}

\prelab

The week before you are to do the lab, briefly familiarize
yourself visually with the apparatus.

\prelabquestion  In the equation $eV_s= hf - E_s $, verify that
all three terms have the same units.

\prelabquestion  Plan how you will analyze your data to determine 
Planck's constant. Note that you don't know $E_s$, so you can't
just write down an equation solved for $h$, with $E_s$ in it.

\analysis

Extract Planck's constant 
from your data, with error bars (see appendix \ref{appendix:graphing}).
Excel has a bug that makes it impossible to graph the data from this
lab (apparently because it involves very big and very small numbers),
so use LibreOffice, not Excel. As with pretty much all computer software,
the right way to enter scientific notation is, for example, 3.4e-12 for $3.4\times10^{-12}$.

Is your result for Planck's constant consistent
with the previously measured value?

Every electron that absorbs a photon acquires a kinetic
energy equal to $hf$. Thus it would seem that if the
voltage is less than the stopping voltage, every electron
should have enough energy to reach the other electrode. Give
two reasons why many electrons do not reach the other
electrode even when the voltage is less than the stopping voltage.
