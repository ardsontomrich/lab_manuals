
\lab{Electromagnetism}\label{lab:electromagnetism}

\apparatus
\equipn{oscilloscope (Tektronix TDS 1001B)}{1/group}
\equipn{microphone (RS 33-1067)}{for 6 groups}
\equipn{microphone (Shure C606)}{for 1 group}
\equipn{PI-9587C sine wave generator}{1/group}
\equip{various tuning forks, mounted on wooden boxes}
\equipn{solenoid (Heath)}{1/group}
\equipn{2-meter wire with banana plugs}{1/group}
\equipn{magnet (stack of 6 Nd)}{1/group}
\equip{masking tape}
\equip{string}

\begin{goals}

\item[] Learn to use an oscilloscope.

\item[] Observe electric fields induced by changing magnetic fields.

\item[] Build a generator.

\item[] Discover Lenz's law.
\end{goals}

\introduction

Physicists hate complication, and when physicist Mich\-ael
Faraday was first learning physics in the early 19th
century, an embarrassingly complex aspect of the science was
the multiplicity of types of forces. Friction, normal
forces, gravity, electric forces, magnetic forces, surface
tension --- the list went on and on. Today, 200 years later,
ask a physicist to enumerate the fundamental forces of
nature and the most likely response will be ``four: gravity,
electromagnetism, the strong nuclear force and the weak
nuclear force.'' Part of the simplification came from the
study of matter at the atomic level, which showed that
apparently unrelated forces such as friction, normal forces,
and surface tension were all manifestations of electrical
forces among atoms. The other big simplification came from
Faraday's experimental work showing that electric and
magnetic forces were intimately related in previously
unexpected ways, so intimately related in fact that we now
refer to the two sets of force-phenomena under a single
term, ``electromagnetism.''

Even before Faraday, Oersted had shown that there was at
least some relationship between electric and magnetic
forces. An electrical current creates a magnetic field, and
magnetic fields exert forces on an electrical current. In
other words, electric forces are forces of charges acting on
charges, and magnetic forces are forces of moving charges on
moving charges. (Even the magnetic field of a bar magnet is
due to currents, the currents created by the orbiting
electrons in its atoms.)

Faraday took Oersted's work a step further, and showed that
the relationship between electricity and magnetism was even
deeper. He showed that a changing electric field produces a
magnetic field, and a changing magnetic field produces an
electric field. Faraday's work forms the basis for such
technologies as the transformer, the electric guitar, the
transformer, and generator, and the electric motor. It also
led to the understanding of light as an electromagnetic wave.

\fig{em-osc-panel}

\mysubsubsection{The Oscilloscope}

An oscilloscope graphs an electrical signal that varies as
a function of time.
The graph is drawn from left to right across the screen,
being painted in real time as the input signal varies.
The purpose of the oscilloscope in this lab is to measure
electromagnetic induction, but to get familiar with the oscillopscope, we'll start out by
using the signal from a microphone as an input,
allowing you to see sound waves. 

The input signal is
supplied in the form of a voltage.
The input connector on the front of the oscilloscope
accepts a type of cable
known as a BNC cable.
A BNC cable is a specific example of coaxial cable
(``coax''), which is also used in cable TV, radio, and
computer networks. The electric current flows in one direction
through the central conductor, and returns in the opposite
direction through the outside conductor, completing the
circuit. The outside conductor is normally kept at ground,
and also serves as shielding against radio interference. 
The advantage of coaxial cable is that it is capable of
transmitting rapidly varying signals without distortion.

\fig{em-osc-coax}

Most of the voltages we wish to measure are not big enough
to use directly for the vertical deflection voltage, so the
oscilloscope actually amplifies the input voltage, i.e., the
small input voltage is used to control a much large voltage
generated internally. The amount of amplification is
controlled with a knob on the front of the scope. For
instance, setting the knob on 1 mV selects an amplification
such that 1 mV at the input deflects the electron beam by
one square of the 1-cm grid. Each 1-cm division is referred
to as a ``division.''

\fig{em-osc-waveform} % \label{fig:em-osc-waveform} is much lower down, to force ref to be on the page where the fig actually lands

\mysubsubsection{The Time Base and Triggering}

Since the $X$ axis represents time, there also has to be a way
to control the time scale, i.e., how fast the imaginary
``penpoint'' sweeps across the screen. For instance, setting
the knob on 10 ms causes it to sweep across one square
in 10 ms. This is known as the time base.

In the figure, suppose the time base is 10 ms.
The scope has 10
divisions, so the total time required for the beam to sweep
from left to right would be 100 ms. This is far too short a
time to allow the user to examine the graph.
The oscilloscope has a built-in method of overcoming this
problem, which works well for periodic (repeating)
signals. The amount of
time required for a periodic signal to perform its pattern
once is called the period. With a periodic signal, all you
really care about seeing what one period or a few periods in
a row look like --- once you've seen one, you've seen them
all. The scope displays one screenful of the
signal, and then keeps on overlaying more and more copies of
the wave on top of the original one. Each
trace is erased when the next one starts, but is being overwritten
continually by later, identical copies of the wave form. You
simply see one persistent trace.\label{fig:em-osc-waveform} % label is much lower down than fig, to force ref to be on the page where the fig actually lands

How does the scope know when to start a new trace? If the
time for one sweep across the screen just happened to be
exactly equal to, say, four periods of the signal, there
would be no problem. But this is unlikely to happen in real
life --- normally the second trace would start from a
different point in the waveform, producing an offset copy of
the wave. Thousands of traces per second would be superimposed
on the screen, each shifted horizontally by a different
amount, and you would only see a blurry band of light.

To make sure that each trace starts from the same point in
the waveform, the scope has a triggering circuit. You use a
knob to set a certain voltage level, the trigger level, at
which you want to start each trace. The scope waits for the
input to move across the trigger level, and then begins a
trace. Once that trace is complete, it pauses until the
input crosses the trigger level again. To make extra sure
that it is really starting over again from the same point in
the waveform, you can also specify whether you want to start
on an increasing voltage or a decreasing voltage ---
otherwise there would always be at least two points in a
period where the voltage crossed your trigger level.

\section*{Setup}


To start with, we'll use a sine wave generator, which makes
a voltage that varies sinusoidally with time. This gives you
a convenient signal to work with while you get the scope working.
Use the black and white outputs on the PI-9587C.

The figure on the last page is a simplified
drawing of the front panel of a digital oscilloscope, showing
only the most important controls you'll need for this lab.
When you turn on the oscilloscope, it will take a while to start
up.

\mysubsubsection{Preliminaries:}

\begin{itemize}
\item[] Press DEFAULT SETUP.

\item[] Use the SEC/DIV knob to
put the time base on something reasonable compared to the
period of the signal you're looking at. The time base is
displayed on the screen, e.g., 10 ms/div, or 1 s/div.

\item[] Use the VOLTS/DIV knob to
put the voltage scale (Y axis) on a reasonable scale
compared to the amplitude of the signal you're looking at.

\item[] The scope has two channels, i.e., it can accept input
through two BNC connectors and display both or either. You'll
only be using channel 1, which is the only one represented in
the simplified drawing. By default, the oscilloscope draws
graphs of both channels' inputs; to get rid of ch. 2, hold
down the CH 2 MENU button (not shown in the diagram) for a couple
of seconds. You also want to make sure that the scope is triggering
on CH 1, rather than CH 2. To do that, press the TRIG MENU
button, and use an option button to select CH 1 as the source.
Set the triggering mode to normal, which is the mode in which
the triggering works as I've described above.
If the trigger level is set to a level that the signal never
actually reaches, you can play with the knob that sets the trigger
level until you get something. A quick and easy way to do this
without trial and error is to use the SET TO 50\%
button, which
automatically sets the trigger level to midway between the top
and bottom peaks of the signal.

\item[] You want to select AC, not DC or GND, on the channel you're using. You
are looking at a voltage that is alternating, creating an
alternating current, ``AC.'' The ``DC'' setting is only
necessary when dealing with constant or very slowly varying
voltages. The ``GND'' simply draws a graph using $y=0$,
which is only useful in certain situations, such as when you
can't find the trace. To select AC, press the CH 1 MENU
button, and select AC coupling.

\end{itemize}

Observe the effect of changing the voltage scale and time
base on the scope. Try changing the frequency and amplitude
on the sine wave generator.

You can freeze the display by pressing RUN/STOP, and then
unfreeze it by pressing the button again.

\section{Preliminary Observations}

Now try observing signals from the microphone.

Notes for the group that uses the Shure mic: As with the Radio Shack
mics, polarity matters. The tip of the phono plug connector is the
live connection, and the part farther back from the tip is the
grounded part. You can connect onto the phono plug with alligator
clips.

Once you have your setup working, try measuring the period
and frequency of the sound from a tuning fork, and make sure
your result for the frequency is the same as what's
written on the tuning fork.
\section{Qualitative Observations}

In this lab you will use a permanent magnet to produce
changing magnetic fields. This causes an electric field to
be induced, which you will detect using a solenoid (spool of
wire) connected to an oscilloscope. The electric field
drives electrons around the solenoid, producing a current
which is detected by the oscilloscope.

Note that although I've described the standard way of triggering
a scope, when the time base is very long, triggering becomes unnecessary.
These scopes are programmed so that when the time base is very long,
they simply continuously display traces.

\labpart{ A constant magnetic field}

Do you detect any signal on the oscilloscope when the magnet
is simply placed at rest inside the solenoid? Try the most
sensitive voltage scale.

\labpart{ A changing magnetic field}

Do you detect any signal when you move the magnet or wiggle
it inside the solenoid or near it? What happens if you
change the speed at which you move the magnet?

\labpart{ Moving the solenoid}

What happens if you hold the magnet still and move the solenoid? 

The poles of the magnet are its flat faces. In later parts
of the lab you will need to know which is north. Determine
this now by hanging it from a string and seeing how it aligns itself with the Earth's field.
The pole that points north is called the north pole of the
magnet. The field pattern funnels into the body of the magnet
through its south pole, and reemerges at its north pole.

\labpart{ A generator}

Tape the magnet securely to the eraser end of a pencil so
that its flat face (one of its two poles) is like the head
of a hammer, and mark the north and south poles of the
magnet for later reference. Spin the pencil near the
solenoid and observe the induced signal. You have built a
generator. (I have unfortunately not had any luck lighting a
lightbulb with the setup, due to the relatively high
internal resistance of the solenoid.)

\section*{Trying Out Your Understanding}

\labpart{ Changing the speed of the generator}

If you change the speed at which you spin the pencil, you
will of course cause the induced signal to have a longer or
shorter period. Does it also have any effect on the
\emph{amplitude} of the wave?

\labpart{ A solenoid with fewer loops}

Use the two-meter cable to make a second solenoid with the
same diameter but fewer loops. Compare the strength of the
induced signals.

\labpart{Dependence on distance}

How does the signal picked up by your generator change with distance?

Try to explain what you have observed, and discuss your
interpretations with your instructor.

\section*{Lenz's Law}

Lenz's law describes how the clockwise or counterclockwise
direction of the induced electric field's whirl\-pool pattern
relates to the changing magnetic field. The main result of
this lab is a determination of how Lenz's law works. To
focus your reasoning, here are four possible forms for Lenz's law:

1. The electric field forms a pattern that is clockwise when
viewed along the direction of the $B$ vector of the
changing magnetic field.

2. The electric field forms a pattern that is counterclockwise
when viewed along the direction of the $B$ vector of the
changing magnetic field.

3. The electric field forms a pattern that is clockwise when
viewed along the direction of the $\Delta B$ vector of the
changing magnetic field.

4. The electric field forms a pattern that is counterclockwise
when viewed along the direction of the $\Delta B$ vector of
the changing magnetic field.

Your job is to figure out which is correct.

The most direct way to figure out Lenz's law is to 
make a tomahawk-chopping motion that ends up with the magnet in the solenoid,
observing whether the pulse induced is positive or negative.
What happens when you reverse the
chopping motion, or when you reverse the north and south
poles of the magnet? Try all four possible combinations and
record your results.

To set up the scope, press DEFAULT SETUP. This should have the
effect of setting the scope on DC coupling, which is what you
want. (If it's on AC coupling, it tries to filter out any DC
part of the input signals, which distorts the results.) To check
that you're on DC coupling, you can do CH 1 MENU, and check that
Coupling says DC. Set the triggering mode ("Mode") to Auto.

Make sure the scope is on DC coupling, not AC coupling, or your
pulses will be distorted.

\fig{em-len-scopepolarity}

It can be tricky to make the connection between the polarity
of the signal on the screen of the oscilloscope and the
direction of the electric field pattern. The figure shows an
example of how to interpret a positive pulse: the current
must have flowed through the scope from the center conductor
of the coax cable to its outer conductor (marked GND on the
coax-to-banana converter). 

\prelab

\prelabquestion  In the sample oscilloscope trace shown on
page \pageref{fig:em-osc-waveform}, what is
the period of the waveform? What is its frequency? The time base is
10 ms.

\prelabquestion  In the same example, again assume the time base is 10
ms/division. The voltage scale is 2 mV/div\-ision. Assume
the zero voltage level is at the middle of the vertical
scale. (The whole graph can actually be shifted up and down
using a knob called ``position.'')  What is the trigger
level currently set to? If the trigger level was changed to
2 mV, what would happen to the trace?

\prelabquestion  Referring to the chapter of your textbook on sound,
which of the following would be a reasonable time base to
use for an audio-frequency signal? 10 ns, 1$\mu$ s, 1 ms, 1 s

\prelabquestion  Does the oscilloscope show you the signal's period,
or its wavelength? Explain.

\prelabquestion The time-scale for all the signals is determined
by the fact that you're wiggling and waving the magnet by hand,
so what's a reasonable order of magnitude to choose for the
time base on the oscilloscope?

\selfcheck

Determine which version of Lenz's law is correct.
