\lab{Relativity}\label{lab:relativity}

\apparatus
\equipn{magnetic balance}{1/group}
\equipn{meter stick}{1/group}
\equipn{multimeter (BK, not HP)}{1/group}
\equipn{laser}{1/group}
\equipn{vernier calipers}{1/group}
\equip{staples}
\equip{DC power supply (Mastek, 30 A)}
\equip{digital balance}
\equip{box of special cables}

\goal{Measure the speed of light.}

\introduction

Oersted discovered that magnetism is an interaction of moving
charges with moving charges, but it wasn't until almost a hundred
years later that Einstein showed why such an interaction must
exist: magnetism occurs as a direct result of his theory of
relativity. Since magnetism is a purely relativistic effect,
and relativistic effects depend on the speed of light, any
measurement of a magnetic effect can be used to determine the
speed of light.

\setup

The idea is to set up opposite currents in two wires, A and B, one under
the other, and use the repulsion between the currents to create an upward force on
the top wire, A. The top wire is on the arm of a balance, which has
a stable equilibrium because of the weight C hanging below it. 
You initially set up the balance with no current through the wires,
adjusting the counterweight D so that the distance between the wires
is as small as possible. What we care about is
really the center-to-center distance (which we'll call $R$),
so even if the wires are almost touching, there's still
a millimeter or two worth of distance between them. By shining a laser at
the mirror, E, and observing the spot it makes on the wall, you can very
accurately determine this particular position of the balance, and tell later
on when you've reproduced it.

\fig{mo-rel-apparatus}

If you put a current through the wires, it will raise wire A. The
torque made by the magnetic repulsion is now canceling the torque made
by gravity directly on all the hardware, such as the masses C and D.
This gravitational torque was zero before, but now you don't know what it
is. The trick is to put a tiny weight (a staple) on top of wire A, and
adjust the current so that the balance returns to the position it originally
had, as determined by the laser dot on the wall. You now know that the
gravitational torque acting on the original apparatus (everything except
for the staple) is back to zero, so the only torques acting are the torque
of gravity on the staple and the magnetic torque. Since both these torques
are applied at the same distance from the axis, the forces creating these
torques must be equal as well. By weighing a block of staples, you can
determine the weight of one staple, and infer the magnetic force that was
acting.

It's very important to get the wires A and B perfectly parallel. You also
need to minimize the resistance of the apparatus, or else you won't be
able to get enough current through it to cancel the weight of the staple.
Most of the resistance is at the polished metal knife-edges that the
moving part of the balance rests on. It may be necessary to clean the
surfaces, or even to freshen them a little with a file to remove any layer
of oxidation. Use the separate BK meter to measure the current --- not the
meter built into the power supply.

The power supply has some strange behavior that makes it not work unless
you power it up in exactly the right way. It has four knobs, going from
left to right: (1) current regulation, (2) over-voltage protection, (3)
fine voltage control, (4) coarse voltage control. Before turning the power
supply on, turn knobs 1 and 2 all the way up, and knobs 3 and 4 all the way
down. Turn the power supply on. Now use knobs 3 and 4 to control how much
current flows.

\pagebreak

\analysis

The first figure below shows a model that explains the repulsion felt by
one of the charges in wire A due to all the charges in wire B. This
is represented in the frame of the lab. For convenience of analysis,
we give the model some unrealistic features: rather than having positively
charged nuclei at rest and negatively charged electrons moving, we pretend
that both are moving, in opposite directions. Since wire B has zero net
density of charge everywhere, it creates no electric fields. (If you like,
you can verify this during lab by putting tiny pieces of paper near the
wires and verifying that they do not feel any static-electrical attraction.)
Since there is no electric field, the force on
the charge in wire A must be purely magnetic.

\fig{mo-rel-lab-frame}

The second figure shows the same scene from the point of view of the
charge in wire A. This charge considers itself to be at rest, and it
also sees the light-colored charges in B as being at rest. In this
frame the dark-colored charges in B are the only ones moving, and they
move with twice the
speed they had in the lab frame. In this frame, the particle in A
is at rest, so it can't feel any magnetic force. The force is now
considered to be purely electric. This electric force exists because
the dark charges are relativistically contracted, which makes them more dense than their
light-colored neighbors, causing a nonzero net density of charge in
wire B. 

\fig{mo-rel-charges-frame}

We've considered the force acting on a single charge in wire A. The
actual force we observe in the experiment is the sum of all the forces
acting on all such charges (of both signs).
As in the slightly different example analyzed in 
m4_ifdef([:__lm:],[:section 23.2 of \emph{Light and Matter}:],[:section 11.1.1 of \emph{Simple Nature}:])%
, this effect is proportional to the product
of the speeds of the charges in the two wires, divided by $c^2$.
Therefore the effect must
be proportional to the product of the currents over $c^2$. In this experiment, the
same current flows through wire A and then comes back through B in the opposite
direction, so we conclude that the force must be proportional to $I^2/c^2$.

In the second frame, the force is purely electrical, and 
m4_ifdef([:__lm:],[:as shown in example 4 in section 22.3 of \emph{Light and Matter}:],[:as can easily be shown by Gauss's law:])%
, the electric field of a charged
wire falls off in proportion to $1/R$, where $R$ is the distance from the
wire. Electrical forces are also proportional to the Coulomb constant $k$.

The longer the wires, the more charges interact, so we must also have a
proportionality to the length $\ell$.

Putting all these factors together, we find that the force
is proportional to $kI^2\ell/c^2R$. We can easily verify that the units of this expression
are newtons, so the only possible missing factor is something unitless. This
unitless factor turns out to be 2 
m4_ifdef([:__lm:],[:--- essentially the same 2 found in example 14 in section 22.7:],[:by Gauss's law:])%
.
The result for the repulsive force between the two wires is
\begin{equation*}
  F = \frac{k}{c^2} \cdot \frac{2I^2\ell}{R} \qquad .
\end{equation*}

By solving this equation you can find $c$.
Your final result is the speed of light, with error bars.

\prelab

\lasersafety

\prelabquestion Show that  $kI^2\ell/c^2R$ has units of newtons.

\prelabquestion Do the algebra to solve for $c$ in terms of the measured quantities.
