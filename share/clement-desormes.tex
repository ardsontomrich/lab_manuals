\lab{The Clement-Desormes Experiment}\label{lab:clement-desormes}

\apparatus
\equip{large flask}
\equip{glass syringe}
\equip{water manometer}
\equip{helium (medium size cylinder, \$40 from Party City)}
\equip{difluoroethane (sold in cans as gas duster at Fry's)}
\equip{stopwatch}
\equip{hose clamps}
\equip{grabber clamps}
\equip{stands}

\introduction

Although the theory that matter was made of atoms started to be talked about
seriously by scientists as early as Galileo's time, scientists generally didn't
think of it as something that was literally true. They considered the atomic
theory to be a useful \emph{model}, but they thought that any fundamental explanation of
real-world phenomena should avoid talking about hypothetical things like atoms.
This feeling was so strong that the physicist Ludwig Boltzmann, who came up with
an atomic explanation of entropy, was driven to suicide by the harsh criticism
to which his ideas were subjected. Even more suspect than the existence of
atoms was any attempt to discuss things like the shapes of molecules that could
be formed by putting them together like tinkertoys; such ideas seemed much too far removed
from the possibility of any experimental testing.

Surprisingly, then, a simple experiment, due to Clem\-ent and Desormes, is capable
of distinguishing two samples of gas that differ only by the \emph{shape} of their
molecules, even if the gases have the same density and are composed of
molecules having the same mass.

\fig{th-clement-desormes}

Use the glass syringe to apply a slight overpressure to the air inside the
flask, causing the difference in height between the water in the
two sides of the manometer to be about 30 cm. Wait one minute to
make sure the air is in thermal equilibrium with the room, and
then take a pressure reading, $p_1$. Release the pressure by popping
the cork for precisely one second, timed on a stopwatch. The air cools
slightly due to its expansion, because it does mechanical work as it
exits throught the valve. However, because the expansion is rapid, and heat
conduction is a slow process, we can treat this as insulated expansion, as discussed
in Appendix 2 of Simple Nature. If the gas is a monoatomic one, such as helium,
then the amount of cooling of the gas, as proved in the book, is given by the
relation $T\propto P^b$, where $b=2/5$. If the gas is not monoatomic, however,
then its molecules can rotate,\footnote{An individual atom in a monoatomic gas has essentially all its
mass concentrated in the nucleus exactly at its center, so it takes an effectively infinite
amount of energy to make it rotate with a certain amount of angular momentum.} and at any given time some of its energy is in the
form of kinetic energy along the $x$, $y$, and $z$ axes, but some is in the form
of rotational kinetic energy. Extracting a given amount of energy from a diatomic
or polyatomic gas, therefore, doesn't cool it as much as it would cool a monoatomic
gas, and it turns out that $b=2/7$ for a diatomic gas, and $1/4$ for a polyatomic gas.\footnote{You'll
often see this stated in terms of the variable $\gamma=1/(1-b)$, which takes on the values
5/3, 7/5, and 4/3.}

Wait one minute for the air to warm back up
to room temperature. The pressure comes back up somewhat as the air warms back
up, and although you should wait a full minute to make sure it's back
in thermal equilibrium, most of the rewarming occurs during the first
few seconds after you finish venting the initial pressure. The pressure
will recover to a value $p_2$ which is less than $p_1$. The ratio
$p_2/p_1$ gives the value of $b$ for the gas.\footnote{In terms of the variable
$\gamma$, we have $\gamma=p_1/(p_1-p_2)$.}

I'm still working on improving this lab.
The basic idea I have in mind is to have you do the lab once with helium (monoatomic),
air (diatomic), and difluoroethane (polyatomic),
and observe the differences in the results due to the different shapes
of the molecules. There are various systematic errors in the experiment, 
so my own absolute results for the $b$ of air haven't been of extremely high
precision; however, in a comparative experiment, I think it will be easy to see
a difference in $b$ between the gases. One possible problem with the air is that it
contains water vapor, which messes up the thermodynamic properties of
the air, because water droplets can condense out of the air when the pressure
is dropped suddenly, as when you open a can of beer. The helium and difluoroethane
shouldn't have this problem. In the spring semester of 2008, we tried all three gases,
and found that it was fairly easy to detect a clear systematic difference
between a higher $b$ for air (.20, .29, .33, .29, and .31 for the five lab groups) and
a lower one for difluoroethane (.18, .20, .33, .25, and .24), but the results for helium
were much lower than theory, and barely distinguishable from air (.29, .35, .35, .31, and .31,
versus .40 according to theory). This may be because we're not actually getting the flasks
as full of pure helium as we think we are.

Some of the flasks have holes at both the top and the bottom. With these flasks, it's
a good idea to introduce the helium through the bottom hole, since it's lighter than
air, and will rise. The difluoroethane, on the other hand, should be put in through the
top hole, because it's heavier than air. I don't know if it will be practical to use
the helium with the flasks that only have holes at the top.

Both the helium and the difluoroethane can displace the beathable air in the classroom,
and the amount of helium in the large canister is particularly big.
For this reason, I've been dispensing the helium outside the classroom.

The difluoroethane is a liquid when it's pressurized inside the can. When you vent
some of the pressure through the nozzle, the pressure drops, and some of it vaporizes
and comes out. The vaporization consumes energy, so the can becomes cold. If you hold
the can upside down and spray it, liquid is emitted rather than gas; this liquid is
extremely cold, and can cause frostbite if it gets on your skin. The gas is not flammable,
and does not harm the ozone layer. Some teenagers have intentionally inhaled it to get
high, so the manufacturers have added a bitterant.
