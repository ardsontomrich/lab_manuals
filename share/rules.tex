\section*{Rules and Organization}

Collection of raw data is work you share with your lab partners.
Once you're done collecting data, you need to do your own analysis.
E.g., it is not okay for two people to turn in the same calculations,
or on a lab requiring a graph for the whole group to make one graph
and turn in copies.

You'll do some labs as formal writeups, others as informal ``check-off''
labs. As described in the syllabus, they're worth different numbers of
points, and you have to do a certain number of each type by the end of
the semester.

The format of formal lab writeups is given in appendix \ref{appendix:format}
on page \pageref{format-really-here}. The raw data section must be contained in your
bound lab notebook. Typically people word-process the abstract section, and any
other sections that don't include much math, and stick the printout in the notebook
to turn it in. The calculations and reasoning section will usually just consist
of hand-written calculations you do in your lab notebook. You need two lab notebooks,
because on days when you turn one in, you need your other one to take raw data in
for the next lab. You may find it convenient to leave one or both of your notebooks
in the cupboard at your lab bench whenever you don't need to have them at home to
work on; this eliminates the problem of forgetting to bring your notebook to school.

For a check-off lab, the main thing I'll pay attention to is your abstract. The rest
of your work for a check-off lab can be informal, and I may not ask to see it unless
I think there's a problem after reading your abstract.
