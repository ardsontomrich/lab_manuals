\lab{The Pendulum}\label{lab:pendulum}

\apparatus
\equip{string}
\equip{cylindrical pendulum bobs}
\equip{hooked masses}
\equip{protractor}
\equip{stopwatch}
\equip{computer with photogate and Vernier Timer software}
\equip{clamps (not hooks) for holding the string}
\equip{tape measures}
\equip{meter sticks}

\goal{Find out how the period of a pendulum depends on its length
and mass, and on the amplitude of its swing.}

\introduction

Until the industrial revolution, the interest of the world's
cultures in the measurement of time was almost entirely
concentrated on the construction of calendars, so that
agricultural cycles could be anticipated.  Although the
Egyptians were the first to divide the day and night into 12
hours, there was no technology for measuring time units
smaller than a day with great accuracy until four thousand years later.

Galileo was the first to realize that a pendulum could be
used to measure time accurately --- previously, he had been
using his own pulse to measure the time required for objects
to roll down inclined planes.  The legend is that the idea
came to him while he watched a chandelier swinging during a
church service.  Sentenced to house arrest for suspicion of
heresy, he spent the last years of his life trying to build
a more practical pendulum clock that would run for long
periods of time without tending.  This technical feat was
only achieved later by Christian Huygens.  Along with the
Chinese invention of the compass, accurate clocks were vital
for European exploration by sea, because longitude can only
be determined by astronomical observations combined with
accurate measurements of time. 

\section*{Notation and Terminology}

When a moving thing, such as a wave, an orbiting planet, a
wheel, or a pendulum, goes through a repetitive cycle of
motion, the time required for one complete cycle is called
the period, $T$.  Note that a pendulum visits any given
point once while traveling in one direction and once while
traveling in the opposite direction.  The period is defined
as how long it takes to come back to the same point,
traveling in the same direction.

\figcaption{vw-pen-period}{From a to g is one full period of the pendulum.
From a to e is not a full period.  Even though
the pendulum has returned at e to its original
position in a, it is moving in the opposite direction,
and has not performed every type of motion it
will ever perform.}
\fig{vw-pen-definitions}

The amplitude of a repetitive motion is a way of describing
the amount of motion.  We can define the amplitude, $A$, of
the pendulum's motion as the maximum angle to which it
rises, i.e., half the total angle swept out.  Let us denote
the mass of the bob, or weight at the end of the pendulum,
by $m$, and the length of the pendulum, from the pivot to
the middle of the weight, as $L$.

\observations

Make observations to determine how the period, $T$, depends
on $A,L$, and $m$.  You will want to use the technique of
isolation of variables.  That means that rather than trying
many random combinations of $A,L$, and $m$, you should keep
two of them constant while measuring $T$ for various values
of the third variable.  Then you should shift your attention
to the next variable, changing it while keeping the other
two constant, and so on.  Be sure to try quite a few values
of the variable you are changing, so you can see in detail
how $T$ depends on each variable.

The period can be measured using the photogate.  See
appendix \ref{appendix:photogate} for how to use the computer software; you want
the mode that's meant specifically for measuring the period
of a pendulum.  Note that the bob is what is blocking the
photogate, so if your bob is irregularly shaped, your
measurements could be messed up if it changed orientation
between one pass through the photogate and the next.  The
easiest way to make sure this problem doesn't occur is to
use a bob with a circular cross-section, so it has the same
width no matter which way the photogate cuts through it.

One of the notable differences between the way students and
professional scientists approach experiments is that
students tend to be timid about exploring extreme conditions.
 In this experiment, there is a big advantage to taking
measurements over wide ranges of each of the three
parameters, because it may be impossible to ascertain how
the period depends on a parameter if you only explore a
small range. When changing $L$, you can go up to four meters
if you hang the pendulum from the balcony; however, you should avoid lengths so short
that they are comparable to the size of the bob itself,
since such short lengths would have anomalous behavior.

For large values of $L$, it's not practical to use the computer, so
use a stopwatch instead. Don't just time one oscillation, because then
the precision of your timing will be horrible. Measure the time required
for some large number of oscillations.

Warning: Since $L$ is measured to the middle of the weight,
you must change the length of the string if you want to vary
$m$ while keeping $L$ constant, compensating for the
different physical size of the new weight.

\prelab

\prelabquestion  What is the maximum possible amplitude for a pendulum of
the type you'll use, whose bob hangs from a string?  If you
were using a pendulum with a stiff rod instead of a string,
you could release it from straight up.  What would its
period be if you could release it from exactly straight up?

\prelabquestion  Referring to appendix 5, how will you
tell from your log-log plot whether the
data follow a power law, i.e., whether it is even appropriate
to try to extract $p$? (If you've already done lab \ref{lab:air-friction},
it's exactly the same technique.)

\selfcheck

Figure out which variable $T$ depends on most strongly, and
extract $p$ (see below).

\analysis

Graph your data and state your conclusions about whether $T$
depends on $A$, $L$ and $m$.  Remember that on a graph of
experimental data, the horizontal axis should always be the
quantity you controlled directly, and the vertical axis
should be the quantity you measured but did not directly
select. The photogate is so accurate that there is not much
point in putting error bars on your graph --- they would be
too small to see. Remember, however, that there are some
fairly significant systematic errors, e.g., it is hard to
accurately keep $L$ the same when switching masses.

It may happen that when you change one of the variables,
there are only small, insignificant changes in the period,
but depending on how you graph the data, it may look like
these are real changes in the period. Most computer graphing
software has a default which is to make the $y$ axis stretch
only across the range of actual $y$ data. e.g., if your
periods were all between 0.567 and 0.574 s, then the
software makes an extremely magnified graph, with the $y$
axis running only over the short range from 0.567 to 0.574
s. On such a scale, it may seem at first glance that there
are some major changes in the period. To help yourself
interpret your graphs, you should make them all with the
same $y$ scale, going from zero all the way up to the
highest period you ever measured. Then you'll be comparing
all three graphs on the same footing.

Of the three variables, find the one on which the period
depends the most strongly, and use the techniques outlined
in appendix \ref{appendix:powerlaws} to see if you can find an equation
describing the relationship between the period and that
variable.  Assume that the equation is of the form

\begin{equation*}
      T  =  cx^p   \qquad   ,  
\end{equation*}

where $x$ would actually be $A,L$ or $m$, and $c$ and $p$
are constants.  The constant $p$ is important, and is
expected to be the same for all pendula.  For instance, if
you find that the mass is the variable that has the greatest
effect on the period, and that the relationship is of the
form $T=cm^3$, then you have discovered something that is
probably generally true for all pendula: that the period is
proportional to the cube of the mass. The constant $c$ is
just some boring number that's not worth extracting from your graphs;
it's the exponent $p$ that's interesting and universally valid.
