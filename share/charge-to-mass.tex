\lab{The Charge to Mass Ratio of the Electron}\label{charge-to-mass}\label{lab:charge-to-mass}



\apparatus
\equip{vacuum tube with Helmholtz}
\equipn{coils (Leybold )}{1}
\equipn{Cenco 33034 HV supply}{1}
\equipn{12-V DC power supplies (Thornton)}{1}
\equipn{multimeters (Fluke or HP)}{2}
\equipn{compass}{1}
\equipn{ruler}{1}
\equip{banana-plug cables}

\goal{Measure the charge-to-mass ratio of the electron.}

\introduction

Why should you believe electrons exist? By the turn of the
twentieth century, not all scientists believed in the
literal reality of atoms, and few could imagine smaller
objects from which the atoms themselves were constructed.
Over two thousand years had elapsed since the Greeks first
speculated that atoms existed based on philosophical
arguments without experimental evidence. During the Middle
Ages in Europe, ``atomism'' had been considered highly
suspect, and possibly heretical. Finally by the Victorian
era, enough evidence had accumulated from chemical
experiments to make a persuasive case for atoms, but
subatomic particles were not even discussed.

If it had taken two millennia to settle the question of
atoms, it is remarkable that another, subatomic level of
structure was brought to light over a period of only about
five years, from 1895 to 1900. Most of the crucial work was
carried out in a series of experiments by J.J. Thomson, who
is therefore often considered the discoverer of the electron.

\fig{em-eqm-setup}

In this lab, you will carry out a variation on a crucial
experiment by Thomson, in which he measured the ratio of the
charge of the electron to its mass, $q/m$. The basic idea is
to observe a beam of electrons in a region of space where
there is an approximately uniform magnetic field, B. The
electrons are emitted perpendicular to the field, and, it
turns out, travel in a circle in a plane perpendicular to
it. The force of the magnetic field on the electrons is
\begin{equation*}
      F  =  qvB \qquad ,  \qquad (1)  
\end{equation*}

directed towards the center of the circle. Their acceleration is
\begin{equation*}
      a  =  \frac{v^2}{r}  \qquad ,  \qquad  (2)  
\end{equation*}
so using $F=ma$, we can write
\begin{equation*}
      qvB  =  \frac{mv^2}{r} \qquad .  \qquad  (3)  
\end{equation*}
If the initial velocity of the electrons is provided by
accelerating them through a voltage difference $V$, they
have a kinetic energy equal to $qV$, so
\begin{equation*}
   \frac{1}{2}mv^2 = qV \qquad . \qquad   (4)
\end{equation*}
From equations 3 and 4, you can determine $q/m$. Note that
since the force of a magnetic field on a moving charged
particle is always perpendicular to the direction of the
particle's motion, the magnetic field can never do any work
on it, and the particle's KE and speed are therefore constant.

You will be able to see where the electrons are going,
because the vacuum tube is filled with a hydrogen gas at a
low pressure. Most electrons travel large distances through
the gas without ever colliding with a hydrogen atom, but a
few do collide, and the atoms then give off blue light,
which you can see. Although I will loosely refer to ``seeing
the beam,'' you are really seeing the light from the
collisions, not the beam of electrons itself. The manufacturer
of the tube has put in just enough gas to make the beam
visible; more gas would make a brighter beam, but would
cause it to spread out and become too broad to measure it precisely.

The field is supplied by an electromagnet consisting of two
circular coils, each with 130 turns of wire (the same on all
the tubes we have). The coils are placed on the same axis,
with the vacuum tube at the center. A pair of coils arranged
in this type of geometry are called Helmholtz coils. Such a
setup provides a nearly uniform field in a large volume of
space between the coils, and that space is more accessible
than the inside of a solenoid.

\mysubsubsection{Safety}

You will use the Cenco high-voltage supply to make a DC
voltage of about 300 $V$. Two things automatically keep this
from being very dangerous: 

\begin{itemize}
\item[] Several hundred DC volts are far less dangerous than a
similar AC voltage. The household AC voltages of 110 and 220
V are more dangerous because AC is more readily conducted by body tissues.

\item[] The HV supply will blow a fuse if too much current flows.
\end{itemize}

\hvsafety


\setup

Before beginning, make sure you do not have any computer
disks near the apparatus, because the magnetic field could erase them.

Heater circuit: As with all vacuum tubes, the cathode is
heated to make it release electrons more easily.
There is a separate low-voltage power supply built into
the high-voltage supply. It has a set of green plugs that, in different
combinations, allow you to get various low voltage values.
Use it to supply 6 V to the terminals
marked ``heater'' on the vacuum tube.
The tube should start to glow.

Electromagnet circuit: Connect the other Thornton power
supply, in series with an ammeter, to the terminals marked
``coil.'' The current from this power supply goes through
both coils to make the magnetic field. Verify that the
magnet is working by using it to deflect a nearby compass.

High-voltage circuit: Leave the Cenco HV supply unplugged.
It is really three HV circuits in one box. You'll be using
the circuit that goes up to 500 V.
Connect it to the
terminals marked ``anode.'' Ask your instructor to check
your circuit. Now plug in the HV supply and turn up the
voltage to 300 V. You should see the electron beam. If you
don't see anything, try it with the lights dimmed.

\observations

Make the necessary observations in order to find $q/m$,
carrying out your plan to deal with the effects of the
Earth's field. The high voltage is supposed to be 300 V,
but to get an accurate measurement of what it really is
you'll need to use a multimeter rather than the poorly
calibrated meter on the front of the high voltage supply.

The beam can be measured accurately by using the glass
rod inside the tube, which has a centimeter scale marked
on it.

Be sure to compute $q/m$ before you leave the lab. That way
you'll know you didn't forget to measure something
important, and that your result is reasonable compared to
the currently accepted value.

There is a glass rod inside the vacuum tube
with a centimeter scale
on it, so you can measure the diameter $d$ of the beam circle
simply by looking at the place where the glowing beam hits
the scale. This is much more accurate than holding a ruler
up to the tube, because it eliminates the parallax error
that would be caused by viewing the beam and the ruler along
a line that wasn't perpendicular to the plane of the beam.
However, the manufacturing process used in making these tubes
(they're probably hand-blown by a glass blower) isn't very
precise, and on many of the tubes you can easily tell by comparison
with the a ruler that, e.g., the 10.0 cm point on the glass rod is
not really 10.0 cm away from the hole from which the beam emerges.
Past students have painstakingly determined the appropriate corrections, $a$,
to add to the observed diameters by the following electrical method.
If you look at your answer to prelab question P1, you'll see that the
product $Br$ is always a fixed quantity in this experiment. It therefore
follows that $Id$ is also supposed to be constant. They measured $I$ and $d$
at two different values of $I$, and determined the correction $a$ that had
to be added to their $d$ values in order to make the two values of $Id$
equal. The results are as follows:

\begin{tabular}{ll}
  \emph{serial number} & $a$ (cm) \\
  98-16 & 0.0 \\
  9849  & 0.0 \\
  99-10 & -0.2 \\
  99-17 & +0.2 \\
  99-56 & +0.3
\end{tabular}

If your apparatus is one that hasn't already had its $a$ determined, then
you should do the necessary measurements to calibrate it.

\prelab

The week before you are to do the lab, briefly familiarize
yourself visually with the apparatus.

\hvsafety


\prelabquestion  Derive an equation for $q/m$ in terms of $V$, $r$ and B.

\prelabquestion  For an electromagnet consisting of a single circular
loop of wire of radius $b$, the field at a point on its
axis, at a distance $z$ from the plane of the loop, is given by
\begin{equation*}
      B = \frac{2\pi kIb^2}{ c^2( b^2+ z^2)^{3/2}} \qquad .
\end{equation*}
Starting from this equation, derive an equation for the
magnetic field at the center of a pair of Helmholtz coils.
Let the number of turns in each coil be $N$ (in our case,
$N=130)$, let their radius be $b$, and let the distance
between them be $h$. (In the actual experiment, the
electrons are never exactly on the axis of the Helmholtz
coils. In practice, the equation you will derive is
sufficiently accurate as an approximation to the actual
field experienced by the electrons.) If you have trouble
with this derivation, see your instructor in his/her office hours.

\prelabquestion  Find the currently accepted value of $q/m$ for the electron.

\prelabquestion  The electrons will be affected by the Earth's magnetic
field, as well as the (larger) field of the coils. Devise a
plan to eliminate, correct for, or at least estimate the
effect of the Earth's magnetic field on your final $q/m$ value.

\prelabquestion  Of the three circuits involved in this experiment, which
ones need to be hooked up with the right polarity, and for
which ones is the polarity irrelevant?

\prelabquestion  What would you infer if you found the beam of electrons
formed a helix rather than a circle?

\analysis

Determine $q/m$, with error bars.

Answer the following questions:

Q1. Thomson started to become convinced during his
experiments that the ``cathode rays'' observed coming from
the cathodes of vacuum tubes were building blocks of atoms
--- what we now call electrons. He then carried out
observations with cathodes made of a variety of metals, and
found that $q/m$ was the same in every case. How would that
observation serve to test his hypothesis?

Q2. Why is it not possible to determine $q$ and $m$
themselves, rather than just their ratio, by observing
electrons' motion in electric or magnetic fields?

Q3. Thomson found that the $q/m$ of an electron was
thousands of times larger than that of ions in electrolysis.
Would this imply that the electrons had more charge? Less
mass? Would there be no way to tell? Explain.
