\lab{Energy in Fields}\label{lab:energy-in-fields}

\apparatus
\equipn{Heath coils}{1/group}
\equipn{0.01 $\mu\zu{F}$ capacitors}{1/group}
\equipn{Daedalon function generator}{1/group}
\equipn{PASCO PI-9587C sine-wave generator}{1/group}
\equipn{oscilloscope}{1/group}

\goal{Observe how the energy content of a field relates
to the field strength.}

\introduction

\figcaption{em-ene-supersimplified}{A simplified version of the circuit.}
The basic idea of this lab is to observe a circuit like the one shown in
the figure above, consisting of a capacitor and a coil of wire (inductor).
Imagine that we first deposit positive and negative
charges on the plates of the capacitor. If we imagined that the universe was
purely mechanical, obeying Newton's laws of motion,
we would expect that the attractive force between these charges
would cause them to come back together and reestablish a stable equilibrium
in which there was zero net charge everywhere in the circuit.

However, the capacitor in its initial, charged, state has an electric 
field between its plates, and this field possesses energy. This energy
can't just go away, because energy is conserved. What really happens is
that as charge starts to flow off of the capacitor plates, a current is
established in the coil. This current creates a magnetic field in the space
inside and around the coil. The electric energy doesn't just evaporate;
it turns into magnetic energy. We end up with an oscillation in which the
capacitor and the coil trade energy back and forth. Your goal is to
monitor this energy exchange, and to use it to deduce a power-law
relationship between each field and its energy.

\figcaption{em-ene-simplified}{The actual circuit.}
The practical realization of the circuit involves some further
complications, as shown in the second figure. 

The wires are not superconductors, so the circuit
has some nonzero resistance, and the oscillations would therefore
gradually die out, as the electric and magnetic energies were
converted to heat. The sine wave generator serves both to
initiate the oscillations and to maintain them, replacing, in each cycle, the
energy that was lost to heat. 

Furthermore, the circuit has a resonant frequency at it
prefers to oscillate, and when the resistance is very small,
the width of the resonance is very narrow. To make the
resonance wider and less finicky, we intentionally insert
a 10 $\zu{k}\Omega$ resistor. The inductance of the coil is about 1 H, which
gives a resonant frequency of about 1.5 kHz.

The actual circuit consists of the 1 H
Heath coil, a 0.01 $\mu F$ capacitance supplied by the
decade capacitor box, a 10 $\zu{k}\Omega$ resistor, and the PASCO sine
wave generator (using the GND and LO $\Omega$ terminals).

\observations
Let $E$ be the magnitude of the electric field between the capacitor plates, and
 let $\tilde{E}$ be the maximum value of this quantity. It is then
 convenient to define
 $x=E/\tilde{E}$, a unitless quantity ranging from $-1$ to 1.
 Similarly, let $y=B/\tilde{B}$ for the corresponding magnetic quantities.
 The electric field is proportional to the voltage difference across
 the capacitor plates, which is something we can measure directly using
 the oscilloscope:
 \begin{equation*}
 	x = \frac{E}{\tilde{E}} = \frac{V_C}{\tilde{V_C}} 
 \end{equation*}

 Magnetic fields are created by moving charges, i.e., by currents.
 Unfortunately, an oscilloscope doesn't measure current, so there's
 no equally direct way to get a handle on the magnetic field. However,
 all the current that goes through the coil must also go through the
 resistor, and Ohm's law relates the current through the resistor to
 the voltage drop across it. This voltage drop is something we can
 measure with the oscilloscope, so we have
 \begin{equation*}
 	y = \frac{B}{\tilde{B}} = \frac{I}{\tilde{I}}= \frac{V_R}{\tilde{V_R}} 
 \end{equation*}
 
 To measure $x$ and $y$, you need to connect channels 1 and 2 of the
 oscilloscope across the resistor and the capacitor. Since both channels
 of the scope are grounded on one side (the side with the ground tab
 on the banana-to-bnc connector), you need to make sure that their
 grounded sides both go to the piece of wire between the resistor and
 the capacitor. Furthermore, one output of the sine wave generator is
 normally grounded, which would mess everything up: two different points
 in the circuit would be grounded, which would mean that there would be
 a short across some of the circuit elements. To avoid this, loosen the
 banana plug connectors on the sine wave generator, and swing away the
 piece of metal that normally connects one of the output plugs to the
 ground.
 
 Tune the sine wave generator's frequency to resonance, and take the
 data you'll need in order to determine $x$ and $y$ at a whole
 bunch of different places over one cycle.

Some of the features of the digital oscilloscopes can make the measurements
a lot easier. Doing Acquire$>$Average tells the scope to average together
a series of up to 128 measurements in order to reduce the amount of noise.
Doing CH 1 MENU$>$Volts/Div$>$Fine allows you to scale the display arbitrarily.
Rather than reading voltages by eye from the scope's x-y grid, you can make
the scope give you a measuring cursor. Do Cursor$>$Type$>$Time. Use the top
left knob to move the cursor to different times. Doing Source$>$CH 1 and
Source$>$CH 2 gives you the voltage measurement for each channel. (Always use
Cursor 1, never Cursor 2.)

The quality of the results can depend a lot on the quality of the connections.
If the display on the scope changes noticeably when you wiggle the wires,
you have a problem.
 
 \analysis
 Plot $y$ versus $x$ on a piece of graph paper. Let's assume that
 the energy in a field depends on the field's strength raised to
 some power $p$. Conservation of energy then gives
 \begin{equation*}
 	|x|^p+|y|^p = 1 \qquad .
 \end{equation*}
 Use your graph to determine $p$, and interpret your result.
 
 \prelab
 
 \prelabquestion
 Sketch what your graph would look like for $p=0.1$, $p=1$, $p=2$, and $p=10$.
 (You should be able to do $p=1$ and $p=2$ without any computations. For
 $p=0.1$ and $p=10$, you can either run some numbers on your calculator or
 use your mathematical knowledge to sketch what they would turn out like.)
