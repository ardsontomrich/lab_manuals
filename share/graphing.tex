\myappendix{graphing}{Graphing}

\section*{Review of Graphing}
Many of your analyses will involve making graphs. A graph
can be an efficient way of presenting data visually,
assuming you include all the information needed by the
reader to interpret it. That means labeling the axes and
indicating the units in parentheses, as in the example. A
title is also helpful. Make sure that distances along the
axes correctly represent the differences in the quantity
being plotted. In the example, it would not have been
correct to space the points evenly in the horizontal
direction, because they were not actually measured at
equally spaced points in time. 

\fig{graph}


\section*{Graphing on a Computer}
Making graphs by hand in your lab notebook is fine, but in
some cases you may find it saves you time to do graphs on a
computer. For computer graphing, I recommend LibreOffice, which
is free, open-source software. It's installed on the computers
in rooms 416 and 418. Because LibreOffice is free, you can download
it and put it on your own computer at home without paying money.
If you already know Excel, it's very similar --- you almost can't
tell it's a different program.

Here's a brief rundown on using LibreOffice:
\begin{itemize}
  \item[] On Windows, go to the Start menu and choose All Programs, LibreOffice, and LibreOffice Calc.
          On Linux, do Applications, Office, OpenOffice, Spreadsheet.
  \item[] Type in your x values in the first column, and your
  	y values in the second column. For scientific notation, do, e.g., 5.2e-7
        to represent $5.2\times10^{-7}$.
  \item[] Select those two columns using the mouse.
  \item[] From the Insert menu, do Chart.
  \item[] When it offers you various styles of graphs to choose from, choose
          the icon that shows a scatter plot, with dots on it  (XY Chart).
  \item[] Adjust the scales so the actual data on the plot is as big as possible,
          eliminating wasted space. To do this, double-click on the graph\footnote{On Windows, single-click.} so that it's
          surrounded by a gray border. Then do Format, Axis, X Axis or Y Axis, Scale.
\end{itemize}
If you want error bars on your graph you can either draw them in by hand or put them in a
separate column of your spreadsheet and doing Insert, Y Error Bars, Cell Range.

\section*{Fitting a Straight Line to a Graph by Hand}

Often in this course you will end up graphing some data
points, fitting a straight line through them with a ruler,
and extracting the slope.

\fig{linearfit}

In this example, panel (a) shows the data, with error bars
on each data point. Panel (b) shows a best fit, drawn by eye
with a ruler. The slope of this best fit line is 100 cm/s.
Note that the slope should be extracted from the line
itself, not from two data points. The line is more reliable
than any pair of individual data points.

In panel (c), a ``worst believable fit'' line has been
drawn, which is as different in slope as possible from the
best fit, while still pretty much staying consistent the
data (going through or close to most of the error bars). Its
slope is 60 cm/s. We can therefore estimate that the
precision of our slope is +40 cm/s.

There is a tendency when drawing a ``worst believable fit''
line to draw instead an ``unbelievably crazy fit'' line, as
in panel (d). The line in panel (d), with a very small
slope, is just not believable compared to the data --- it is
several standard deviations away from most of the data points.

\section*{Fitting a Straight Line to a Graph on a Computer}
It's also possible to fit a straight line to a graph using computer
software such as LibreOffice. 

To do this, double-click on your graph\footnote{On Windows, single-click.} so that it's outlined in gray.
Go to the Insert menu, choose
Trend Lines,\footnote{``Trend line'' is scientifically illiterate terminology
that originates from Microsoft Office, which LibreOffice slavishly copies. If you
don't want to come off as an ignoramus, call it a ``fit'' or ``line of best fit.''
} choose Linear, and check the box for Show equation.

How accurate is your slope? A method for getting error bars on the slope
is to artificially change one of your data
points to reflect your estimate of how much it could have been off,
and then redo the fit and find the new slope. The change in the slope
tells you the error in the slope that results from the error in this
data-point. You can then repeat this for the other points and
proceed as in appendix \ref{appendix:errpropagation}. In some cases, such as the absolute zero
lab and the photoelectric effect lab, it's very hard to tell how accurate
your raw data are \emph{a priori}; in these labs, you can use the typical
amount of deviation of the points from the line as an estimate of their
accuracy.
