\myappendix{powerlaws}{Finding Power Laws from Data}
For many people, it is hard to imagine how scientists
originally came up with all the equations that can now be
found in textbooks.  This appendix explains one method for
finding equations to describe data from an experiment.

\section*{Linear and nonlinear relationships}
When two variables $x$ and $y$ are related by an equation of the form
\begin{equation*}
		y  =  cx   \qquad   ,  
\end{equation*}
where $c$ is a constant (does not depend on $x$ or $y)$, we
say that a linear relationship exists between $x$ and $y$. 
As an example, a harp has many strings of different lengths
which are all of the same thickness and made of the same
material.  If the mass of a string is $m$ and its length is
$L$, then the equation
\begin{equation*}
		m  =  cL  
\end{equation*}
will hold, where $c$ is the mass per unit length, with units of kg/m.
Many quantities in the physical world are instead related in
a nonlinear fashion, i.e., the relationship does not fit the
above definition of linearity.  For instance, the mass of a
steel ball bearing is related to its diameter by an
equation of the form
\begin{equation*}
		m  =  cd^3   \qquad   ,  
\end{equation*}
where $c$ is the mass per unit volume, or density, of steel.
 Doubling the diameter does not double the mass, it
increases it by a factor of eight.

\section*{Power laws}
Both examples above are of the general mathematical form
\begin{equation*}
		y  =  cx^p   \qquad   ,  
\end{equation*}
which is known as a power law.  In the case of a linear
relationship, $p=1$.  Consider the (made-up) experimental
data shown in the table.

\begin{tabular}{lp{30mm}p{30mm}}
			& $h$=height of rodent at the shoulder (cm) & $f$=food eaten per day (g) \\
\hline
	shrew	&	1	& 3\\
	rat		&	10	& 300\\
	capybara	& 100	& 30,000
\end{tabular}

It's fairly easy to figure out what's going on just by
staring at the numbers a little. Every time you increase the
height of the animal by a factor of 10, its food consumption
goes up by a factor of 100. This implies that $f$ must be
proportional to the square of $h$, or, displaying the
proportionality constant $k=3$ explicitly,
\begin{equation*}
	f=3h^2   \qquad   .
\end{equation*}

\section*{Use of logarithms}
Now we have found $c=3$ and $p=2$ by inspection, but that
would be much more difficult to do if these weren't all
round numbers. A more generally applicable method to use
when you suspect a power-law relationship is to take
logarithms of both variables. It doesn't matter at all what
base you use, as long as you use the same base for both
variables. Since the data above were increasing by powers of
10, we'll use logarithms to the base 10, but personally I
usually just use natural logs for this kind of thing.

\begin{tabular}{lp{30mm}p{30mm}}
			& $\log_{10} h$ & $\log_{10} f$ \\
\hline
	shrew	&	0	& 0.48\\
	rat		&	1	& 2.48\\
	capybara	& 2	& 4.48
\end{tabular}


This is a big improvement, because differences are so much
simpler to work mentally with than ratios. The difference
between each successive value of $h$ is 1, while $f$ increases
by 2 units each time. The fact that the logs of the $f's$
increase twice as quickly is the same as saying that $f$ is
proportional to the square of $h$.

\section*{Log-log plots}

Even better, the logarithms can be interpreted visually
using a graph, as shown on the next page. The slope of this type of
log-log graph gives the power $p$. Although it is also
possible to extract the proportionality constant, $c$, from
such a graph, the proportionality constant is usually much
less interesting than $p$. For instance, we would suspect
that if $p=2$ for rodents, then it might also equal 2 for
frogs or ants. Also, $p$ would be the same regardless of
what units we used to measure the variables. The constant
$c$, however, would be different if we used different units,
and would also probably be different for other types of animals.

\fig{rodents}
